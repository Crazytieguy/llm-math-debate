\author{Yoav Tzfati}
\documentclass{article}
\usepackage{graphicx}
\usepackage{amsfonts}
\usepackage{amsmath}

\setlength{\parindent}{0pt}
\linespread{1.15}
\title{llm math debate WolframAlpha formatted examples}

\begin{document}

\maketitle

Complete the square:

$y^2-y-5 x^2+x-8=0$

\hrule

1. Add $8$ to both sides:

$y^2-y-5 x^2+x=8$

2. Group terms with $x\text{ and }y$ separately, leaving placeholder constants:

$\left(-5 x^2+x+\underline{\text{   }}\right)+\left(y^2-y+\underline{\text{   }}\right)=\underline{\text{   }}+8$

3. $\left(-5 x^2+x+\underline{\text{   }}\right)=-5 \left(x^2-\frac{x}{5}+\underline{\text{   }}\right):$

$-5 \left(x^2-\frac{x}{5}+\underline{\text{   }}\right)+\left(y^2-y+\underline{\text{   }}\right)=\underline{\text{   }}+8$

4. $\begin{array}{l}
\text{Take one half of the coefficient of }x\text{ and square it. Then add it to both sides of the equation, multiplying by the factored constant }-5\text{ on the right. }\\
\text{ Insert }\left(\frac{\frac{-1}{5}}{2}\right)^2=\frac{1}{100}\text{ on the left and }\frac{-5}{100}=-\frac{1}{20}\text{ on the right}: \\
\end{array}$

$-5 \left(x^2-\frac{x}{5}+\frac{1}{100}\right)+\left(y^2-y+\underline{\text{   }}\right)=-\frac{1}{20}+8$

5. $8-\frac{1}{20}=\frac{159}{20}$:

$-5 \left(x^2-\frac{x}{5}+\frac{1}{100}\right)+\left(y^2-y+\underline{\text{   }}\right)=\frac{159}{20}$

6. $\begin{array}{l}
\text{Take one half of the coefficient of }y\text{ and square it, then add it to both sides. }\\
\text{ Add }\left(\frac{-1}{2}\right)^2=\frac{1}{4}\text{ to both sides}: \\
\end{array}$

$-5 \left(x^2-\frac{x}{5}+\frac{1}{100}\right)+\left(y^2-y+\frac{1}{4}\right)=\frac{1}{4}+\frac{159}{20}$

7. $\frac{159}{20}+\frac{1}{4}=\frac{41}{5}$:

$-5 \left(x^2-\frac{x}{5}+\frac{1}{100}\right)+\left(y^2-y+\frac{1}{4}\right)=\frac{41}{5}$

8. $x^2-\frac{x}{5}+\frac{1}{100}=\left(x-\frac{1}{10}\right)^2:$

$-5 \left(x-\frac{1}{10}\right)^2+\left(y^2-y+\frac{1}{4}\right)=\frac{41}{5}$

9. $y^2-y+\frac{1}{4}=\left(y-\frac{1}{2}\right)^2:$

Answer: $-5 \left(x-\frac{1}{10}\right)^2+\left(y-\frac{1}{2}\right)^2=\frac{41}{5}$

\pagebreak

Factor the following:

$\frac{45 x}{2}-9 x^2$

\hrule

1. Put each term in $\frac{45 x}{2}-9 x^2\text{ over the common denominator }2: \frac{45 x}{2}-9 x^2\text{ = }\frac{45 x}{2}-\frac{18 x^2}{2}$:

$\frac{45 x}{2}-\frac{18 x^2}{2}$

2. $\frac{45 x}{2}-\frac{18 x^2}{2}=\frac{45 x-18 x^2}{2}$:

$\frac{45 x-18 x^2}{2}$

3. Factor $-9 x\text{ out of }45 x-18 x^2:$

Answer: $\frac{-9 x (2 x-5)}{2}$

\pagebreak

Expand the following:

$\left(-\frac{26 x^2}{e}-\frac{14 x}{e}+\frac{27}{e}\right) \left(\frac{6 x}{e}+\frac{3}{e}\right)$

\hrule

1. $\left(-\frac{26 x^2}{e}-\frac{14 x}{e}+\frac{27}{e}\right) \left(\frac{6 x}{e}+\frac{3}{e}\right)=\frac{3 \left(-\frac{26 x^2}{e}-\frac{14 x}{e}+\frac{27}{e}\right)}{e}+\frac{6 x \left(-\frac{26 x^2}{e}-\frac{14 x}{e}+\frac{27}{e}\right)}{e}$:

Answer: $\frac{3 \left(-\frac{26 x^2}{e}-\frac{14 x}{e}+\frac{27}{e}\right)}{e}+\frac{6 x \left(-\frac{26 x^2}{e}-\frac{14 x}{e}+\frac{27}{e}\right)}{e}$

\pagebreak

Solve for $x:$

$-11 x^2-13 x+5=0$

\hrule

1. Divide both sides by $-11:$

$x^2+\frac{13 x}{11}-\frac{5}{11}=0$

2. Add $\frac{5}{11}$ to both sides:

$x^2+\frac{13 x}{11}=\frac{5}{11}$

3. Add $\frac{169}{484}$ to both sides:

$x^2+\frac{13 x}{11}+\frac{169}{484}=\frac{389}{484}$

4. Write the left hand side as a square:

$\left(x+\frac{13}{22}\right)^2=\frac{389}{484}$

5. Take the square root of both sides:

$x+\frac{13}{22}=\frac{\sqrt{389}}{22}\text{ or }x+\frac{13}{22}=-\frac{\sqrt{389}}{22}$

6. Subtract $\frac{13}{22}$ from both sides:

$x=\frac{\sqrt{389}}{22}-\frac{13}{22}\text{ or }x+\frac{13}{22}=-\frac{\sqrt{389}}{22}$

7. Subtract $\frac{13}{22}$ from both sides:

Answer: $x=\frac{\sqrt{389}}{22}-\frac{13}{22}\text{ or }x=-\frac{13}{22}-\frac{\sqrt{389}}{22}$

\pagebreak

Simplify the following:

$\sqrt{61}+\sqrt{125}$

\hrule

1. $\sqrt{125}\text{ = }\sqrt{5^3}\text{ = }5 \sqrt{5}$:

Answer: $\sqrt{61}+5 \sqrt{5}$

\pagebreak

Solve for $x:$

$\sqrt{4 x+5}+\sqrt{14 x+11}=13$

\hrule

1. $\left(\sqrt{4 x+5}+\sqrt{14 x+11}\right)^2=16+18 x+2 \sqrt{4 x+5} \sqrt{14 x+11}=16+18 x+2 \sqrt{(4 x+5) (14 x+11)}=169:$

$16+18 x+2 \sqrt{(4 x+5) (14 x+11)}=169$

2. Subtract $18 x+16$ from both sides:

$2 \sqrt{(4 x+5) (14 x+11)}=153-18 x$

3. Raise both sides to the power of two:

$4 (4 x+5) (14 x+11)=(153-18 x)^2$

4. Expand out terms of the left hand side:

$224 x^2+456 x+220=(153-18 x)^2$

5. Expand out terms of the right hand side:

$224 x^2+456 x+220=324 x^2-5508 x+23409$

6. Subtract $324 x^2-5508 x+23409$ from both sides:

$-100 x^2+5964 x-23189=0$

7. Divide both sides by $-100:$

$x^2-\frac{1491 x}{25}+\frac{23189}{100}=0$

8. Subtract $\frac{23189}{100}$ from both sides:

$x^2-\frac{1491 x}{25}=-\frac{23189}{100}$

9. Add $\frac{2223081}{2500}$ to both sides:

$x^2-\frac{1491 x}{25}+\frac{2223081}{2500}=\frac{410839}{625}$

10. Write the left hand side as a square:

$\left(x-\frac{1491}{50}\right)^2=\frac{410839}{625}$

11. Take the square root of both sides:

$x-\frac{1491}{50}=\frac{13 \sqrt{2431}}{25}\text{ or }x-\frac{1491}{50}=-\frac{13 \sqrt{2431}}{25}$

12. Add $\frac{1491}{50}$ to both sides:

$x=\frac{1491}{50}+\frac{13 \sqrt{2431}}{25}\text{ or }x-\frac{1491}{50}=-\frac{13 \sqrt{2431}}{25}$

13. Add $\frac{1491}{50}$ to both sides:

$x=\frac{1491}{50}+\frac{13 \sqrt{2431}}{25}\text{ or }x=\frac{1491}{50}-\frac{13 \sqrt{2431}}{25}$

14. $\sqrt{4 x+5}+\sqrt{14 x+11}\text{ $\approx $ }13.:$

So this solution is correct

15. $\sqrt{4 x+5}+\sqrt{14 x+11}\text{ $\approx $ }43.1221:$

So this solution is incorrect

16. The solution is:

Answer: $x=\frac{1491}{50}-\frac{13 \sqrt{2431}}{25}$

\pagebreak

Solve the following system:

$\{
        \begin{array}{l}
-9 \sqrt{3} y-\sqrt{3} x+4 \sqrt{3}=0  \\
            -11 \sqrt{3} y+4 \sqrt{3} x-\sqrt{3}=0 \\
        \end{array}$

\hrule

1. Express the system in standard form:

$\{
        \begin{array}{l}
-9 \sqrt{3} y-\sqrt{3} x=-4 \sqrt{3} \\
            -11 \sqrt{3} y+4 \sqrt{3} x=\sqrt{3} \\
        \end{array}$

2. Express the system in matrix form:

$\left(
        \begin{array}{cc}
-\sqrt{3}  & -9 \sqrt{3}  \\
                4 \sqrt{3} & -11 \sqrt{3} \\
            \end{array}
        \right)\left(
        \begin{array}{c}
x \\
                y \\
            \end{array}
        \right)\text{ = }\left(
        \begin{array}{c}
-4 \sqrt{3} \\
                \sqrt{3}    \\
            \end{array}
        \right)$

3. Solve the system with Cramer's rule:

$x=\frac{
            \begin{array}{|c|c|}
-4 \sqrt{3} & -9 \sqrt{3}  \\
                \sqrt{3}    & -11 \sqrt{3} \\
            \end{array}
        }{
            \begin{array}{|c|c|}
-\sqrt{3}  & -9 \sqrt{3}  \\
                4 \sqrt{3} & -11 \sqrt{3} \\
            \end{array}
        }\text{ and }y=\frac{
            \begin{array}{|c|c|}
-\sqrt{3}  & -4 \sqrt{3} \\
                4 \sqrt{3} & \sqrt{3}    \\
            \end{array}
        }{
            \begin{array}{|c|c|}
-\sqrt{3}  & -9 \sqrt{3}  \\
                4 \sqrt{3} & -11 \sqrt{3} \\
            \end{array}
        }$

4. Evaluate the determinant 

$\begin{array}{|c|c|}
-\sqrt{3}  & -9 \sqrt{3}  \\
            4 \sqrt{3} & -11 \sqrt{3} \\
        \end{array}
        =141:$

$x=\frac{
            \begin{array}{|c|c|}
-4 \sqrt{3} & -9 \sqrt{3}  \\
                \sqrt{3}    & -11 \sqrt{3} \\
            \end{array}
        }{141}\text{ and }y=\frac{
            \begin{array}{|c|c|}
-\sqrt{3}  & -4 \sqrt{3} \\
                4 \sqrt{3} & \sqrt{3}    \\
            \end{array}
        }{141}$

5. Evaluate the determinant 

$\begin{array}{|c|c|}
-4 \sqrt{3} & -9 \sqrt{3}  \\
            \sqrt{3}    & -11 \sqrt{3} \\
        \end{array}
        =159:$

$x=\frac{159}{141}\text{ and }y=\frac{
            \begin{array}{|c|c|}
-\sqrt{3}  & -4 \sqrt{3} \\
                4 \sqrt{3} & \sqrt{3}    \\
            \end{array}
        }{141}$

6. The gcd of 159 and 141 is $3,\text{ so }\frac{159}{141}=\frac{53\times 3}{47\times 3}=\frac{53}{47}$:

$x=\frac{53}{47}\text{ and }y=\frac{
            \begin{array}{|c|c|}
-\sqrt{3}  & -4 \sqrt{3} \\
                4 \sqrt{3} & \sqrt{3}    \\
            \end{array}
        }{141}$

7. Evaluate the determinant 

$\begin{array}{|c|c|}
-\sqrt{3}  & -4 \sqrt{3} \\
            4 \sqrt{3} & \sqrt{3}    \\
        \end{array}
        =45:$

$x=\frac{53}{47}\text{ and }y=\frac{45}{141}$

8. The gcd of 45 and 141 is $3,\text{ so }\frac{45}{141}=\frac{15\times 3}{47\times 3}=\frac{15}{47}$:

Answer: $x=\frac{53}{47}\text{ and }y=\frac{15}{47}$

\pagebreak

Possible derivation:

$\frac{d}{dx}\left(e^{-3-2 x}+\sqrt[3]{6-\frac{x}{2}}\right)$

\hrule

1. Differentiate the sum term by term:

= $\frac{d}{dx}\left(e^{-3-2 x}\right)+\frac{d}{dx}\left(\sqrt[3]{6-\frac{x}{2}}\right)$

2. Using the chain rule, $\frac{d}{dx}\left(e^{-2 x-3}\right)=\frac{de^u}{du} \frac{du}{dx},\text{ where }u=-2 x-3\text{ and }\frac{d\text{}}{du}\left(e^u\right)=e^u:$

= $\frac{d}{dx}\left(\sqrt[3]{6-\frac{x}{2}}\right)+e^{-3-2 x} \left(\frac{d}{dx}(-3-2 x)\right)$

3. Differentiate the sum term by term and factor out constants:

= $\frac{d}{dx}\left(\sqrt[3]{6-\frac{x}{2}}\right)+\frac{d}{dx}(-3)-2 \left(\frac{d}{dx}(x)\right) e^{-3-2 x}$

4. The derivative of $-3$ is zero:

= $\frac{d}{dx}\left(\sqrt[3]{6-\frac{x}{2}}\right)+e^{-3-2 x} \left(-2 \left(\frac{d}{dx}(x)\right)+0\right)$

5. Simplify the expression:

= $\frac{d}{dx}\left(\sqrt[3]{6-\frac{x}{2}}\right)-2 e^{-3-2 x} \left(\frac{d}{dx}(x)\right)$

6. Using the chain rule, $\frac{d}{dx}\left(\sqrt[3]{6-\frac{x}{2}}\right)=\frac{d\sqrt[3]{u}}{du} \frac{du}{dx},\text{ where }u=6-\frac{x}{2}\text{ and }\frac{d\text{}}{du}\left(\sqrt[3]{u}\right)=\frac{1}{3 u^{2/3}}$:

= $-2 e^{-3-2 x} \left(\frac{d}{dx}(x)\right)+\frac{\frac{d}{dx}\left(6-\frac{x}{2}\right)}{3 \left(6-\frac{x}{2}\right)^{2/3}}$

7. Differentiate the sum term by term and factor out constants:

= $-2 e^{-3-2 x} \left(\frac{d}{dx}(x)\right)+\frac{d}{dx}(6)-\frac{1}{2} \left(\frac{d}{dx}(x)\right) \frac{1}{3 \left(6-\frac{x}{2}\right)^{2/3}}$

8. The derivative of $6$ is zero:

= $-2 e^{-3-2 x} \left(\frac{d}{dx}(x)\right)+\frac{-\frac{1}{2} \left(\frac{d}{dx}(x)\right)+0}{3 \left(6-\frac{x}{2}\right)^{2/3}}$

9. Simplify the expression:

= $-2 e^{-3-2 x} \left(\frac{d}{dx}(x)\right)-\frac{\frac{d}{dx}(x)}{6 \left(6-\frac{x}{2}\right)^{2/3}}$

10. The derivative of $x\text{ is }1:$

= $-2 e^{-3-2 x} \left(\frac{d}{dx}(x)\right)-1 \frac{1}{6 \left(6-\frac{x}{2}\right)^{2/3}}$

11. The derivative of $x\text{ is }1:$

= $-\frac{1}{6 \left(6-\frac{x}{2}\right)^{2/3}}-1 2 e^{-3-2 x}$

12. Simplify the expression:

= $-2 e^{-3-2 x}-\frac{1}{6 \left(6-\frac{x}{2}\right)^{2/3}}$

13. Simplify the expression:

Answer: = $-2 e^{-3-2 x}-\frac{1}{6 \sqrt[3]{6-\frac{x}{2}}^2}$

\pagebreak

Find the geometric mean of the list:

$(3,729,729,5)$

\hrule

1. The geometric mean of a list of numbers is given by:

$(\text{product of elements})^{\frac{1}{(\text{number of elements})}}$

2. $(\text{product of elements})\text{ = }3\text{ $\times $ 729 $\times $ 729 $\times $ }5:$

$\sqrt[(\text{number of elements})]{(3\text{ $\times $ 729 $\times $ 729 $\times $ }5)}$

3. Counting, we see that the list has $4$ elements:

$(3\text{ $\times $ 729 $\times $ 729 $\times $ }5)^{\frac{1}{4}}\text{ = }\sqrt[4]{3\text{ $\times $ 729 $\times $ 729 $\times $ }5}$

4. $3\text{ $\times $ 729 $\times $ 729 $\times $ }5=7971615:$

$\sqrt[4]{7971615}$

5. $\sqrt[4]{7971615}\text{ = }\sqrt[4]{3^{13}\times 5}\text{ = }3^3 \sqrt[4]{3} \sqrt[4]{5}$:

$3^3 \sqrt[4]{3} \sqrt[4]{5}$

6. $3^3=3\times 3^2:$

$3\times 3^2 \sqrt[4]{3} \sqrt[4]{5}$

7. $3^2=9:$

$3\times 9 \sqrt[4]{3} \sqrt[4]{5}$

8. $3\times 9\text{ = }27:$

$27 \sqrt[4]{3} \sqrt[4]{5}$

9. $\sqrt[4]{3} \sqrt[4]{5}=\sqrt[4]{3\times 5}$:

$27 \sqrt[4]{3\times 5}$

10. $3\times 5\text{ = }15:$

Answer: $27 \sqrt[4]{15}$

\pagebreak

Find the mean of the list:

$(8,-5)$

\hrule

1. The mean of a list of numbers is given by:

$\frac{(\text{sum of elements})}{(\text{number of elements})}$

2. $(\text{sum of elements})\text{ = }8-5:$

$\frac{8-5}{(\text{number of elements})}$

3. Counting, we see that the list has $2$ elements:

$\frac{8-5}{2}$

4. $8-5\text{ = }3:$

Answer: $\frac{3}{2}\approx 1.5$

\pagebreak

Find the median of the list:

$\left(\frac{3}{4},-1,\frac{3}{\sqrt{2}},-3,\frac{25}{3},8,-4 \log (2),-\frac{4}{3},3\right)\text{ = }\left(\frac{3}{4},-1,3\ 2^{-1/2},-3,\frac{25}{3},8,-4 \log (2),-\frac{4}{3},3\right)$

\hrule

1. The list, sorted from smallest to largest, is:

$\left(-3,-4 \log (2),-\frac{4}{3},-1,\frac{3}{4},3\ 2^{-1/2},3,8,\frac{25}{3}\right)$

2. The median of the list $\left(-3,-4 \log (2),-\frac{4}{3},-1,\frac{3}{4},3\ 2^{-1/2},3,8,\frac{25}{3}\right)$ is the element in the middle, which is:

Answer: $\frac{3}{4}$

\pagebreak

Find the mode (commonest element) of the list:

$\left(-\frac{19}{2},10,-\frac{19}{2},\frac{19}{2},0,10,\frac{19}{2},\frac{19}{2},\frac{19}{2},10,0,10,4,0,4,10,10,4,10,4,10,10,\frac{19}{2},10,4,-\frac{19}{2},10,4\right)$

\hrule

1. $\left(-\frac{19}{2},10,-\frac{19}{2},\frac{19}{2},0,10,\frac{19}{2},\frac{19}{2},\frac{19}{2},10,0,10,4,0,4,10,10,4,10,4,10,10,\frac{19}{2},10,4,-\frac{19}{2},10,4\right)=\left(-\frac{19}{2},10,-\frac{19}{2},\frac{19}{2},0,10,\frac{19}{2},\frac{19}{2},\frac{19}{2},10,0,10,4,0,4,10,10,4,10,4,10,10,\frac{19}{2},10,4,-\frac{19}{2},10,4\right):$

$\left(-\frac{19}{2},10,-\frac{19}{2},\frac{19}{2},0,10,\frac{19}{2},\frac{19}{2},\frac{19}{2},10,0,10,4,0,4,10,10,4,10,4,10,10,\frac{19}{2},10,4,-\frac{19}{2},10,4\right)$

2. The sorted list is:

$\left(-\frac{19}{2},-\frac{19}{2},-\frac{19}{2},0,0,0,4,4,4,4,4,4,\frac{19}{2},\frac{19}{2},\frac{19}{2},\frac{19}{2},\frac{19}{2},10,10,10,10,10,10,10,10,10,10,10\right)$

3. The tally of each element is:

$\begin{array}{cccccc}
\text{element }& -\frac{19}{2} & 0 & 4 & \frac{19}{2} & 10 \\
\text{ tally }& 3 & 3 & 6 & 5 & 11 \\
\end{array}$

4. The element 10 appears 11 times. Thus the mode of the list is:

Answer: $10$

\pagebreak

Find the range of the list:

$\left(0,-1,-\frac{33}{10},-\frac{5}{2},3,-\frac{311}{32},5,-2,-5,-\frac{28}{3},-3\right)\text{ = }\left(0,-1,-\frac{33}{10},-\frac{5}{2},3,-\frac{311}{32},5,-2,-5,-\frac{28}{3},-3\right)$

\hrule

1. The range of a list is given by:

$(\text{largest element})-(\text{smallest element})$

2. $\left(0,-1,-\frac{33}{10},-\frac{5}{2},3,-\frac{311}{32},5,-2,-5,-\frac{28}{3},-3\right)\text{  We see that the largest element is }5:$

$5-(\text{smallest element})$

3. $\left(0,-1,-\frac{33}{10},-\frac{5}{2},3,-\frac{311}{32},5,-2,-5,-\frac{28}{3},-3\right)\text{  We see that the smallest element is }-\frac{311}{32}$:

$-\frac{-311}{32}+5$

4. $-\frac{-311}{32}+5=\frac{471}{32}$:

Answer: $\frac{471}{32}$

\pagebreak

Find the (sample) variance of the list:

$(-6,-15,-1,-11)$

\hrule

1. The (sample) variance of a list of numbers $X=\left\{X_1,X_2,\ldots ,X_n\right\}\text{ with mean }\mu =\frac{X_1+X_2+\ldots +X_n}{n}$ is given by:

$\frac{\left| X_1-\mu \right| {}^2+\left| X_2-\mu \right| {}^2+\ldots +\left| X_n-\mu \right| {}^2}{n-1}$

2. There are $n=4\text{ elements in the list }X=\{-6,-15,-1,-11\}$:

$\frac{\left| X_1-\mu \right| {}^2+\left| X_2-\mu \right| {}^2+\left| X_3-\mu \right| {}^2+\left| X_4-\mu \right| {}^2}{4-1}$

3. $\begin{array}{l}
\text{The elements }X_i\text{ of the list }X=\{-6,-15,-1,-11\}\text{ are: }\\
 
\begin{array}{l}
X_1=-6 \\
 X_2=-15 \\
 X_3=-1 \\
 X_4=-11 \\
\end{array}
 \\
\end{array}$

$\frac{| -\mu -6| ^2+| -\mu -15| ^2+| -\mu -1| ^2+| -\mu -11| ^2}{4-1}$

4. The mean $(\mu )\text{ is given by$\backslash $n$\backslash $n}\mu =\frac{X_1+X_2+X_3+X_4}{4}=\frac{-6-15-1-11}{4}=-\frac{33}{4}$:

$\frac{\left| -\frac{-33}{4}-6\right| ^2+\left| -\frac{-33}{4}-15\right| ^2+\left| -\frac{-33}{4}-1\right| ^2+\left| -\frac{-33}{4}-11\right| ^2}{4-1}$

5. $\begin{array}{l}
\text{The values of the differences are: }\\
 
\begin{array}{l}
-\frac{-33}{4}-6=\frac{9}{4} \\
 -\frac{-33}{4}-15=-\frac{27}{4} \\
 -\frac{-33}{4}-1=\frac{29}{4} \\
 -\frac{-33}{4}-11=-\frac{11}{4} \\
 4-1=3 \\
\end{array}
 \\
\end{array}$

$\frac{\left| \frac{9}{4}\right| ^2+\left| -\frac{27}{4}\right| ^2+\left| \frac{29}{4}\right| ^2+\left| -\frac{11}{4}\right| ^2}{3}$

6. $\begin{array}{l}
\text{The values of the terms in the numerator are}: \\
 
\begin{array}{l}
\left| \frac{9}{4}\right| ^2\text{ = }\frac{81}{16} \\
 \left| -\frac{27}{4}\right| ^2\text{ = }\frac{729}{16} \\
 \left| \frac{29}{4}\right| ^2\text{ = }\frac{841}{16} \\
 \left| -\frac{11}{4}\right| ^2\text{ = }\frac{121}{16} \\
\end{array}
 \\
\end{array}$

$\frac{\frac{81}{16}+\frac{729}{16}+\frac{841}{16}+\frac{121}{16}}{3}$

7. $\frac{81}{16}+\frac{729}{16}+\frac{841}{16}+\frac{121}{16}=\frac{443}{4}$:

$\frac{\frac{443}{4}}{3}$

8. $\frac{\frac{443}{4}}{3}=\frac{443}{12}$:

Answer: $\frac{443}{12}$

\pagebreak

Simplify the following:

$\left(
\begin{array}{cc}
-\frac{19}{6} & \frac{53}{6} \\
 \frac{13}{6} & -\frac{2}{3} \\
\end{array}
\right)+\left(
\begin{array}{cc}
\frac{10}{3} & -\frac{1}{2} \\
 2 & 9 \\
\end{array}
\right)$

\hrule

1. $\left(
\begin{array}{cc}
-\frac{19}{6} & \frac{53}{6} \\
 \frac{13}{6} & -\frac{2}{3} \\
\end{array}
\right)+\left(
\begin{array}{cc}
\frac{10}{3} & -\frac{1}{2} \\
 2 & 9 \\
\end{array}
\right)=\left(
\begin{array}{cc}
-\frac{19}{6}+\frac{10}{3} & \frac{53}{6}-\frac{1}{2} \\
 \frac{13}{6}+2 & -\frac{2}{3}+9 \\
\end{array}
\right):$

$\left(
\begin{array}{cc}
-\frac{19}{6}+\frac{10}{3} & \frac{53}{6}-\frac{1}{2} \\
 \frac{13}{6}+2 & -\frac{2}{3}+9 \\
\end{array}
\right)$

2. Put $-\frac{19}{6}+\frac{10}{3}\text{ over the common denominator }6. -\frac{19}{6}+\frac{10}{3}\text{ = }\frac{-19}{6}+\frac{2\times 10}{6}$:

$\left(
\begin{array}{cc}
\frac{-19}{6}+\frac{2\times 10}{6} & \frac{53}{6}-\frac{1}{2} \\
 \frac{13}{6}+2 & -\frac{2}{3}+9 \\
\end{array}
\right)$

3. $2\times 10\text{ = }20:$

$\left(
\begin{array}{cc}
\frac{-19}{6}+\frac{20}{6} & \frac{53}{6}-\frac{1}{2} \\
 \frac{13}{6}+2 & -\frac{2}{3}+9 \\
\end{array}
\right)$

4. $-\frac{19}{6}+\frac{20}{6}\text{ = }\frac{-19+20}{6}$:

$\left(
\begin{array}{cc}
\frac{-19+20}{6} & \frac{53}{6}-\frac{1}{2} \\
 \frac{13}{6}+2 & -\frac{2}{3}+9 \\
\end{array}
\right)$

5. $-19+20=1:$

$\left(
\begin{array}{cc}
\frac{1}{6} & \frac{53}{6}-\frac{1}{2} \\
 \frac{13}{6}+2 & -\frac{2}{3}+9 \\
\end{array}
\right)$

6. Put $\frac{53}{6}-\frac{1}{2}\text{ over the common denominator }6. \frac{53}{6}-\frac{1}{2}\text{ = }\frac{53}{6}-\frac{3}{6}$:

$\left(
\begin{array}{cc}
\frac{1}{6} & \frac{53}{6}-\frac{3}{6} \\
 \frac{13}{6}+2 & -\frac{2}{3}+9 \\
\end{array}
\right)$

7. $\frac{53}{6}-\frac{3}{6}\text{ = }\frac{53-3}{6}$:

$\left(
\begin{array}{cc}
\frac{1}{6} & \frac{53-3}{6} \\
 \frac{13}{6}+2 & -\frac{2}{3}+9 \\
\end{array}
\right)$

8. $\begin{array}{c}
\begin{array}{ccc}
\text{ }& 5 & 3 \\
 - &\text{  }& 3 \\
\hline
\text{  }& 5 & 0 \\
\end{array}
 \\
\end{array}$

:

$\left(
\begin{array}{cc}
\frac{1}{6} & \frac{50}{6} \\
 \frac{13}{6}+2 & -\frac{2}{3}+9 \\
\end{array}
\right)$

9. The gcd of 50 and $6\text{ is }2,\text{ so }\frac{50}{6}=\frac{2\times 25}{2\times 3}=\frac{2}{2}\times \frac{25}{3}=\frac{25}{3}$:

$\left(
\begin{array}{cc}
\frac{1}{6} & \frac{25}{3} \\
 \frac{13}{6}+2 & -\frac{2}{3}+9 \\
\end{array}
\right)$

10. Put $\frac{13}{6}+2\text{ over the common denominator }6. \frac{13}{6}+2\text{ = }\frac{13}{6}+\frac{6\times 2}{6}$:

$\left(
\begin{array}{cc}
\frac{1}{6} & \frac{25}{3} \\
 \frac{13}{6}+\frac{6\times 2}{6} & -\frac{2}{3}+9 \\
\end{array}
\right)$

11. $6\times 2\text{ = }12:$

$\left(
\begin{array}{cc}
\frac{1}{6} & \frac{25}{3} \\
 \frac{13}{6}+\frac{12}{6} & -\frac{2}{3}+9 \\
\end{array}
\right)$

12. $\frac{13}{6}+\frac{12}{6}\text{ = }\frac{13+12}{6}$:

$\left(
\begin{array}{cc}
\frac{1}{6} & \frac{25}{3} \\
 \frac{13+12}{6} & -\frac{2}{3}+9 \\
\end{array}
\right)$

13. $\begin{array}{c}
\begin{array}{ccc}
\hline
\text{  }& 1 & 3 \\
\hline
 + & 1 & 2 \\
\text{  }& 2 & 5 \\
\end{array}
 \\
\end{array}$

:

$\left(
\begin{array}{cc}
\frac{1}{6} & \frac{25}{3} \\
 \frac{25}{6} & -\frac{2}{3}+9 \\
\end{array}
\right)$

14. Put $-\frac{2}{3}+9\text{ over the common denominator }3. -\frac{2}{3}+9\text{ = }\frac{-2}{3}+\frac{3\times 9}{3}$:

$\left(
\begin{array}{cc}
\frac{1}{6} & \frac{25}{3} \\
 \frac{25}{6} & \frac{-2}{3}+\frac{3\times 9}{3} \\
\end{array}
\right)$

15. $3\times 9\text{ = }27:$

$\left(
\begin{array}{cc}
\frac{1}{6} & \frac{25}{3} \\
 \frac{25}{6} & \frac{-2}{3}+\frac{27}{3} \\
\end{array}
\right)$

16. $-\frac{2}{3}+\frac{27}{3}\text{ = }\frac{-2+27}{3}$:

$\left(
\begin{array}{cc}
\frac{1}{6} & \frac{25}{3} \\
 \frac{25}{6} & \frac{-2+27}{3} \\
\end{array}
\right)$

17. $-2+27=25:$

Answer: $\left(
\begin{array}{cc}
\frac{1}{6} & \frac{25}{3} \\
 \frac{25}{6} & \frac{25}{3} \\
\end{array}
\right)$

\pagebreak

Find the characteristic polynomial of the matrix $M\text{ with respect to the variable }\lambda$ :

$M=\left(
    \begin{array}{ccc}
0  & 6  & -1 \\
        -4 & -9 & -5 \\
        -7 & 1  & -2 \\
      \end{array}
    \right)$

\hrule

1. To find the characteristic polynomial of a matrix, subtract a variable multiplied by the identity matrix and take the determinant:

$| M-\lambda  \mathbb{I}|$

2. $\begin{array}{lll}
| M-\lambda  \mathbb{I}| & = & \left|
      \begin{array}{ccc}
0  & 6  & -1 \\
        -4 & -9 & -5 \\
        -7 & 1  & -2 \\
      \end{array}
      -\lambda
      \begin{array}{ccc}
1 & 0 & 0 \\
        0 & 1 & 0 \\
        0 & 0 & 1 \\
      \end{array}
      \right|                               \\
\text{                        }& = & \left|
      \begin{array}{ccc}
0  & 6  & -1 \\
        -4 & -9 & -5 \\
        -7 & 1  & -2 \\
      \end{array}
      -
      \begin{array}{ccc}
\lambda & 0       & 0       \\
        0       & \lambda & 0       \\
        0       & 0       & \lambda \\
      \end{array}
      \right|                               \\
    \end{array}$

= $\left|
    \begin{array}{ccc}
-\lambda & 6           & -1          \\
      -4       & -\lambda -9 & -5          \\
      -7       & 1           & -\lambda -2 \\
    \end{array}
    \right|$

3. Row $3$ has as many or more ones than the others:

= $\left|
    \begin{array}{ccc}
-\lambda & 6           & -1          \\
      -4       & -\lambda -9 & -5          \\
      -7       & 1           & -\lambda -2 \\
    \end{array}
    \right|$

4. The determinant of the matrix $\left(
    \begin{array}{ccc}
-\lambda & 6           & -1          \\
        -4       & -\lambda -9 & -5          \\
        -7       & 1           & -\lambda -2 \\
      \end{array}
    \right)\text{ is given by }(-7)\, \left|
    \begin{array}{cc}
6           & -1 \\
      -\lambda -9 & -5 \\
    \end{array}
    \right| +(-1)\, \left|
    \begin{array}{cc}
-\lambda & -1 \\
      -4       & -5 \\
    \end{array}
    \right| +(-\lambda -2) \left|
    \begin{array}{cc}
-\lambda & 6           \\
      -4       & -\lambda -9 \\
    \end{array}
    \right|$ :

= $(-7)\, \left|
    \begin{array}{cc}
6           & -1 \\
      -\lambda -9 & -5 \\
    \end{array}
    \right| +(-1)\, \left|
    \begin{array}{cc}
-\lambda & -1 \\
      -4       & -5 \\
    \end{array}
    \right| +(-\lambda -2) \left|
    \begin{array}{cc}
-\lambda & 6           \\
      -4       & -\lambda -9 \\
    \end{array}
    \right|$

5. $(-7)\, \left|
    \begin{array}{cc}
6           & -1 \\
      -\lambda -9 & -5 \\
    \end{array}
    \right| =(-7)\, (6 (-5)-(-1)\, -\lambda -9)=-7 (-\lambda -39)=-7 (-\lambda -39)$:

= $-7 (-\lambda -39)+(-1)\, \left|
    \begin{array}{cc}
-\lambda & -1 \\
      -4       & -5 \\
    \end{array}
    \right| +(-\lambda -2) \left|
    \begin{array}{cc}
-\lambda & 6           \\
      -4       & -\lambda -9 \\
    \end{array}
    \right|$

6. $(-1)\, \left|
    \begin{array}{cc}
-\lambda & -1 \\
      -4       & -5 \\
    \end{array}
    \right| =(-1)\, (5 \lambda -(-1)\, (-4))=-(5 \lambda -4)=4-5 \lambda $:

= $-7 (-\lambda -39)+4-5 \lambda +(-\lambda -2) \left|
    \begin{array}{cc}
-\lambda & 6           \\
      -4       & -\lambda -9 \\
    \end{array}
    \right|$

7. $(-\lambda -2) \left|
    \begin{array}{cc}
-\lambda & 6           \\
      -4       & -\lambda -9 \\
    \end{array}
    \right| =(-\lambda -2) ((-\lambda ) (-\lambda -9)-6 (-4))=(-\lambda -2) \left(\lambda ^2+9 \lambda +24\right)=(-\lambda -2) \left(\lambda ^2+9 \lambda +24\right)$:

= $-7 (-\lambda -39)+(4-5 \lambda )+(-\lambda -2) \left(\lambda ^2+9 \lambda +24\right)$

8. $-7 (-\lambda -39)+(4-5 \lambda )+(-\lambda -2) \left(\lambda ^2+9 \lambda +24\right)\, =\, -\lambda ^3-11 \lambda ^2-40 \lambda +229:$

Answer: = $-\lambda ^3-11 \lambda ^2-40 \lambda +229$

\pagebreak

Compute the following cross product:

$\, (-9,-6,5)\, \times \, (-3,5,4)\,$

\hrule

1. Construct a matrix where the first row contains unit vectors $\hat{\text{i}}, \hat{\text{j}},\text{ and }\hat{\text{k}};\text{ and the second and third rows are made of vectors }\, (-9,-6,5)\,\text{  and }\, (-3,5,4)\,$ :

$\left(
\begin{array}{ccc}
\hat{\text{i}} & \hat{\text{j}} & \hat{\text{k}} \\
 -9 & -6 & 5 \\
 -3 & 5 & 4 \\
\end{array}
\right)$

2. Take the determinant of this matrix:

$\left| 
\begin{array}{ccc}
\hat{\text{i}} & \hat{\text{j}} & \hat{\text{k}} \\
 -9 & -6 & 5 \\
 -3 & 5 & 4 \\
\end{array}
\right|$

3. Expand with respect to row $1:$

= $\left| 
\begin{array}{ccc}
\hat{\text{i}} & \hat{\text{j}} & \hat{\text{k}} \\
 -9 & -6 & 5 \\
 -3 & 5 & 4 \\
\end{array}
\right|$

4. The determinant of the matrix $\left(
\begin{array}{ccc}
\hat{\text{i}} & \hat{\text{j}} & \hat{\text{k}} \\
 -9 & -6 & 5 \\
 -3 & 5 & 4 \\
\end{array}
\right)\text{ is given by }\hat{\text{i}} \left| 
\begin{array}{cc}
-6 & 5 \\
 5 & 4 \\
\end{array}
\right| +\left(-\hat{\text{j}}\right) \left| 
\begin{array}{cc}
-9 & 5 \\
 -3 & 4 \\
\end{array}
\right| +\hat{\text{k}} \left| 
\begin{array}{cc}
-9 & -6 \\
 -3 & 5 \\
\end{array}
\right|$ :

= $\hat{\text{i}} \left| 
\begin{array}{cc}
-6 & 5 \\
 5 & 4 \\
\end{array}
\right| +\left(-\hat{\text{j}}\right) \left| 
\begin{array}{cc}
-9 & 5 \\
 -3 & 4 \\
\end{array}
\right| +\hat{\text{k}} \left| 
\begin{array}{cc}
-9 & -6 \\
 -3 & 5 \\
\end{array}
\right|$

5. $\hat{\text{i}} \left| 
\begin{array}{cc}
-6 & 5 \\
 5 & 4 \\
\end{array}
\right| =\hat{\text{i}} ((-6)\, \times \, 4-5\ 5)=\hat{\text{i}} (-49)=-49 \hat{\text{i}}$:

= $-49 \hat{\text{i}}+\left(-\hat{\text{j}}\right) \left| 
\begin{array}{cc}
-9 & 5 \\
 -3 & 4 \\
\end{array}
\right| +\hat{\text{k}} \left| 
\begin{array}{cc}
-9 & -6 \\
 -3 & 5 \\
\end{array}
\right|$

6. $-\hat{\text{j}} \left| 
\begin{array}{cc}
-9 & 5 \\
 -3 & 4 \\
\end{array}
\right| =-\hat{\text{j}} ((-9)\, \times \, 4-5 (-3))=-\hat{\text{j}} (-21)=21 \hat{\text{j}}$:

= $-49 \hat{\text{i}}+21 \hat{\text{j}}+\hat{\text{k}} \left| 
\begin{array}{cc}
-9 & -6 \\
 -3 & 5 \\
\end{array}
\right|$

7. $\hat{\text{k}} \left| 
\begin{array}{cc}
-9 & -6 \\
 -3 & 5 \\
\end{array}
\right| =\hat{\text{k}} ((-9)\, \times \, 5-(-6)\, (-3))=\hat{\text{k}} (-63)=-63 \hat{\text{k}}$:

= $-49 \hat{\text{i}}+21 \hat{\text{j}}+-63 \hat{\text{k}}$

8. Order the terms in a more natural way:

= $-49 \hat{\text{i}}+21 \hat{\text{j}}-63 \hat{\text{k}}$

9. $-49 \hat{\text{i}}+21 \hat{\text{j}}-63 \hat{\text{k}}=\, (-49,21,-63)\,$ :

Answer: $\, (-49,21,-63)\,$

\pagebreak

Find the determinant:

$\left| 
\begin{array}{cc}
-2 & -5 \\
 2 & -4 \\
\end{array}
\right|$

\hrule

1. Multiply along the diagonals and subtract:

$(-2)\, (-4)-(-5)\, \times \, 2$

2. $-2 (-4)\text{ = }8:$

$8--5\times 2$

3. $-(-5)\text{ = }5:$

$8+5\times 2$

4. $5\times 2\text{ = }10:$

$8+10$

5. $8+10=18:$

Answer: $18$

\pagebreak

Take the dot product of the following vectors:

$\, \left(-\frac{2}{e},\frac{1}{e},-\frac{26}{e},\frac{6}{e},\frac{10}{e},\frac{13}{e}\right)\, .\, \left(-\frac{5}{e},\frac{23}{e},-\frac{7}{e},-\frac{8}{e},\frac{19}{e},-\frac{15}{e}\right)\,$

\hrule

1. Multiply the row vector with the column vector:

$\, \left(-\frac{2}{e},\frac{1}{e},-\frac{26}{e},\frac{6}{e},\frac{10}{e},\frac{13}{e}\right)\, .\left(
\begin{array}{c}
-\frac{5}{e} \\
 \frac{23}{e} \\
 -\frac{7}{e} \\
 -\frac{8}{e} \\
 \frac{19}{e} \\
 -\frac{15}{e} \\
\end{array}
\right)$

2. $\left(
\begin{array}{cccccc}
-\frac{2}{e} & \frac{1}{e} & -\frac{26}{e} & \frac{6}{e} & \frac{10}{e} & \frac{13}{e} \\
\end{array}
\right).\left(
\begin{array}{c}
-\frac{5}{e} \\
 \frac{23}{e} \\
 -\frac{7}{e} \\
 -\frac{8}{e} \\
 \frac{19}{e} \\
 -\frac{15}{e} \\
\end{array}
\right):$

$-\frac{2 (-5)}{e e}=\frac{10}{e^2}$

3. $\left(
\begin{array}{cccccc}
-\frac{2}{e} & \frac{1}{e} & -\frac{26}{e} & \frac{6}{e} & \frac{10}{e} & \frac{13}{e} \\
\end{array}
\right).\left(
\begin{array}{c}
-\frac{5}{e} \\
 \frac{23}{e} \\
 -\frac{7}{e} \\
 -\frac{8}{e} \\
 \frac{19}{e} \\
 -\frac{15}{e} \\
\end{array}
\right):$

$\frac{23}{e e}=\frac{23}{e^2}$

4. $\left(
\begin{array}{cccccc}
-\frac{2}{e} & \frac{1}{e} & -\frac{26}{e} & \frac{6}{e} & \frac{10}{e} & \frac{13}{e} \\
\end{array}
\right).\left(
\begin{array}{c}
-\frac{5}{e} \\
 \frac{23}{e} \\
 -\frac{7}{e} \\
 -\frac{8}{e} \\
 \frac{19}{e} \\
 -\frac{15}{e} \\
\end{array}
\right):$

$-\frac{26 (-7)}{e e}=\frac{182}{e^2}$

5. $\left(
\begin{array}{cccccc}
-\frac{2}{e} & \frac{1}{e} & -\frac{26}{e} & \frac{6}{e} & \frac{10}{e} & \frac{13}{e} \\
\end{array}
\right).\left(
\begin{array}{c}
-\frac{5}{e} \\
 \frac{23}{e} \\
 -\frac{7}{e} \\
 -\frac{8}{e} \\
 \frac{19}{e} \\
 -\frac{15}{e} \\
\end{array}
\right):$

$\frac{6 (-8)}{e e}=-\frac{48}{e^2}$

6. $\left(
\begin{array}{cccccc}
-\frac{2}{e} & \frac{1}{e} & -\frac{26}{e} & \frac{6}{e} & \frac{10}{e} & \frac{13}{e} \\
\end{array}
\right).\left(
\begin{array}{c}
-\frac{5}{e} \\
 \frac{23}{e} \\
 -\frac{7}{e} \\
 -\frac{8}{e} \\
 \frac{19}{e} \\
 -\frac{15}{e} \\
\end{array}
\right):$

$\frac{10\ 19}{e e}=\frac{190}{e^2}$

7. $\left(
\begin{array}{cccccc}
-\frac{2}{e} & \frac{1}{e} & -\frac{26}{e} & \frac{6}{e} & \frac{10}{e} & \frac{13}{e} \\
\end{array}
\right).\left(
\begin{array}{c}
-\frac{5}{e} \\
 \frac{23}{e} \\
 -\frac{7}{e} \\
 -\frac{8}{e} \\
 \frac{19}{e} \\
 -\frac{15}{e} \\
\end{array}
\right):$

$\frac{13 (-15)}{e e}=-\frac{195}{e^2}$

8. $\left(-\frac{2}{e}\right)\, \left(-\frac{5}{e}\right)+\frac{23}{e e}+\left(-\frac{26}{e}\right)\, \left(-\frac{7}{e}\right)+\frac{6 (-8)}{e e}+\frac{10\ 19}{e e}+\frac{13 (-15)}{e e}$:

$\frac{10}{e^2}+\frac{23}{e^2}+\frac{182}{e^2}-\frac{48}{e^2}+\frac{190}{e^2}-\frac{195}{e^2}$

9. Add like terms. $\frac{10}{e^2}+\frac{23}{e^2}+\frac{182}{e^2}-\frac{48}{e^2}+\frac{190}{e^2}-\frac{195}{e^2}\text{ = }\frac{162}{e^2}$:

Answer: $\frac{162}{e^2}$

\pagebreak

Find all the eigenvalues of the matrix $M:$

$M=\left(
\begin{array}{cc}
\frac{42}{5} & -4 \\
 -\frac{9}{5} & 1 \\
\end{array}
\right)$

\hrule

1. Find $\lambda \in \mathbb{C}\text{ such that }M v=\lambda  v\text{ for some nonzero vector }v:$

$M v=\lambda  v$

2. Rewrite the equation $M v=\lambda  v\text{ as }(M-\mathbb{I} \lambda ) v=0:$

$(M-\mathbb{I} \lambda ) v=0$

3. Find all $\lambda\text{  such that }| M-\mathbb{I} \lambda | =0:$

$| M-\mathbb{I} \lambda | =0$

4. $\begin{array}{lll}
M-\mathbb{I} \lambda  & = & \left(
\begin{array}{cc}
\frac{42}{5} & -4 \\
 -\frac{9}{5} & 1 \\
\end{array}
\right)-\left(
\begin{array}{cc}
1 & 0 \\
 0 & 1 \\
\end{array}
\right) \lambda  \\
\text{  }& = & \left(
\begin{array}{cc}
\frac{42}{5}-\lambda  & -4 \\
 -\frac{9}{5} & 1-\lambda  \\
\end{array}
\right) \\
\end{array}$

$\left| 
\begin{array}{cc}
\frac{42}{5}-\lambda  & -4 \\
 -\frac{9}{5} & 1-\lambda  \\
\end{array}
\right| =0$

5. $\left| 
\begin{array}{cc}
\frac{42}{5}-\lambda  & -4 \\
 -\frac{9}{5} & 1-\lambda  \\
\end{array}
\right| =\lambda ^2-\frac{47 \lambda }{5}+\frac{6}{5}$:

$\lambda ^2-\frac{47 \lambda }{5}+\frac{6}{5}=0$

6. Subtract $\frac{6}{5}$ from both sides:

$\lambda ^2-\frac{47 \lambda }{5}=-\frac{6}{5}$

7. Add $\frac{2209}{100}$ to both sides:

$\lambda ^2-\frac{47 \lambda }{5}+\frac{2209}{100}=\frac{2089}{100}$

8. Write the left hand side as a square:

$\left(\lambda -\frac{47}{10}\right)^2=\frac{2089}{100}$

9. Take the square root of both sides:

$\lambda -\frac{47}{10}=\frac{\sqrt{2089}}{10}\text{ or }\lambda -\frac{47}{10}=-\frac{\sqrt{2089}}{10}$

10. Add $\frac{47}{10}$ to both sides:

$\lambda =\frac{47}{10}+\frac{\sqrt{2089}}{10}\text{ or }\lambda -\frac{47}{10}=-\frac{\sqrt{2089}}{10}$

11. Add $\frac{47}{10}$ to both sides:

Answer: $\lambda =\frac{47}{10}+\frac{\sqrt{2089}}{10}\text{ or }\lambda =\frac{47}{10}-\frac{\sqrt{2089}}{10}$

\pagebreak

Find all the eigenvalues and eigenvectors of the matrix $M:$

$M=\left(
\begin{array}{cc}
9 & -5 \\
 1 & -4 \\
\end{array}
\right)$

\hrule

1. Find $\lambda \in \mathbb{C}\text{ such that }M v=\lambda  v\text{ for some nonzero vector }v:$

$M v=\lambda  v$

2. Rewrite the equation $M v=\lambda  v\text{ as }(M-\mathbb{I} \lambda ) v=0:$

$(M-\mathbb{I} \lambda ) v=0$

3. Find all $\lambda\text{  such that }| M-\mathbb{I} \lambda | =0:$

$| M-\mathbb{I} \lambda | =0$

4. $\begin{array}{lll}
M-\mathbb{I} \lambda  & = & \left(
\begin{array}{cc}
9 & -5 \\
 1 & -4 \\
\end{array}
\right)-\left(
\begin{array}{cc}
1 & 0 \\
 0 & 1 \\
\end{array}
\right) \lambda  \\
\text{  }& = & \left(
\begin{array}{cc}
9-\lambda  & -5 \\
 1 & -\lambda -4 \\
\end{array}
\right) \\
\end{array}$

$\left| 
\begin{array}{cc}
9-\lambda  & -5 \\
 1 & -\lambda -4 \\
\end{array}
\right| =0$

5. $\left| 
\begin{array}{cc}
9-\lambda  & -5 \\
 1 & -\lambda -4 \\
\end{array}
\right| =\lambda ^2-5 \lambda -31:$

$\lambda ^2-5 \lambda -31=0$

6. Add 31 to both sides:

$\lambda ^2-5 \lambda =31$

7. Add $\frac{25}{4}$ to both sides:

$\lambda ^2-5 \lambda +\frac{25}{4}=\frac{149}{4}$

8. Write the left hand side as a square:

$\left(\lambda -\frac{5}{2}\right)^2=\frac{149}{4}$

9. Take the square root of both sides:

$\lambda -\frac{5}{2}=\frac{\sqrt{149}}{2}\text{ or }\lambda -\frac{5}{2}=-\frac{\sqrt{149}}{2}$

10. Add $\frac{5}{2}$ to both sides:

$\lambda =\frac{5}{2}+\frac{\sqrt{149}}{2}\text{ or }\lambda -\frac{5}{2}=-\frac{\sqrt{149}}{2}$

11. Add $\frac{5}{2}$ to both sides:

$\lambda =\frac{5}{2}+\frac{\sqrt{149}}{2}\text{ or }\lambda =\frac{5}{2}-\frac{\sqrt{149}}{2}$

12. Find all $v\text{ such that }(M-\mathbb{I} \lambda ) v=0\text{ for some eigenvalue }\lambda$ :

$(M-\mathbb{I} \lambda ) v=0$

13. Substitute $\left(
\begin{array}{cc}
9-\lambda  & -5 \\
 1 & -\lambda -4 \\
\end{array}
\right)\text{ for }(M-\mathbb{I} \lambda ):$

$\left(
\begin{array}{cc}
9-\lambda  & -5 \\
 1 & -\lambda -4 \\
\end{array}
\right) v=0$

14. Write $v\text{ as }\left(
\begin{array}{c}
v_1 \\
 v_2 \\
\end{array}
\right)\text{ and }0\text{ as }\left(
\begin{array}{c}
0 \\
 0 \\
\end{array}
\right):$

$\left(
\begin{array}{cc}
9-\lambda  & -5 \\
 1 & -\lambda -4 \\
\end{array}
\right).\left(
\begin{array}{c}
v_1 \\
 v_2 \\
\end{array}
\right)=\left(
\begin{array}{c}
0 \\
 0 \\
\end{array}
\right)$

15. First, substitute $\frac{5}{2}+\frac{\sqrt{149}}{2}\text{ for }\lambda\text{  in the matrix }\left(
\begin{array}{cc}
9-\lambda  & -5 \\
 1 & -\lambda -4 \\
\end{array}
\right)$ and solve the system:

$\left(
\begin{array}{cc}
\frac{13}{2}-\frac{\sqrt{149}}{2} & -5 \\
 1 & -\frac{13}{2}-\frac{\sqrt{149}}{2} \\
\end{array}
\right).\left(
\begin{array}{c}
v_1 \\
 v_2 \\
\end{array}
\right)=\left(
\begin{array}{c}
0 \\
 0 \\
\end{array}
\right)$

16. In augmented matrix form, the system is written as:

$\left(
\begin{array}{ccc}
\frac{13}{2}-\frac{\sqrt{149}}{2} & -5 & 0 \\
 1 & -\frac{13}{2}-\frac{\sqrt{149}}{2} & 0 \\
\end{array}
\right)$

17. Swap row $1\text{ with row }2:$

$\left(
\begin{array}{ccc}
1 & -\frac{13}{2}-\frac{\sqrt{149}}{2} & 0 \\
 \frac{13}{2}-\frac{\sqrt{149}}{2} & -5 & 0 \\
\end{array}
\right)$

18. Subtract $\left(\frac{13}{2}-\frac{\sqrt{149}}{2}\right)\, \times \,\text{ (row }1)\text{ from row }2:$

$\left(
\begin{array}{ccc}
1 & -\frac{13}{2}-\frac{\sqrt{149}}{2} & 0 \\
 0 & 0 & 0 \\
\end{array}
\right)$

19. Translated back to a matrix equation, the reduced system $\left(
\begin{array}{ccc}
1 & -\frac{13}{2}-\frac{\sqrt{149}}{2} & 0 \\
 0 & 0 & 0 \\
\end{array}
\right)$ is:

$\left(
\begin{array}{cc}
1 & -\frac{13}{2}-\frac{\sqrt{149}}{2} \\
 0 & 0 \\
\end{array}
\right)\left(
\begin{array}{c}
v_1 \\
 v_2 \\
\end{array}
\right)=\left(
\begin{array}{c}
0 \\
 0 \\
\end{array}
\right)$

20. As a scalar equation, the system $\left(
\begin{array}{cc}
1 & -\frac{13}{2}-\frac{\sqrt{149}}{2} \\
 0 & 0 \\
\end{array}
\right)\left(
\begin{array}{c}
v_1 \\
 v_2 \\
\end{array}
\right)=\left(
\begin{array}{c}
0 \\
 0 \\
\end{array}
\right)$ translates to:

$v_1+\left(-\frac{\sqrt{149}}{2}-\frac{13}{2}\right) v_2=0$

21. Rewrite the equation as:

$v_1=-\left(-\frac{13}{2}-\frac{\sqrt{149}}{2}\right) v_2$

22. According to the above equation:

$v\text{ = }\left(
\begin{array}{c}
v_1 \\
 v_2 \\
\end{array}
\right)\text{ = }\left(
\begin{array}{c}
-\left(-\frac{13}{2}-\frac{\sqrt{149}}{2}\right) v_2 \\
 v_2 \\
\end{array}
\right)$

23. Letting $v_2=1\text{ in }\left(
\begin{array}{c}
-\left(-\frac{13}{2}-\frac{\sqrt{149}}{2}\right) v_2 \\
 v_2 \\
\end{array}
\right),\text{ we find that }\left(
\begin{array}{c}
\frac{13}{2}+\frac{\sqrt{149}}{2} \\
 1 \\
\end{array}
\right)\text{ is an eigenvector of the matrix }\left(
\begin{array}{cc}
9 & -5 \\
 1 & -4 \\
\end{array}
\right)\text{ associated with the eigenvalue }\frac{5}{2}+\frac{\sqrt{149}}{2}$:

$v=\left(
\begin{array}{c}
\frac{13}{2}+\frac{\sqrt{149}}{2} \\
 1 \\
\end{array}
\right)$

24. Substitute $\frac{5}{2}-\frac{\sqrt{149}}{2}\text{ for }\lambda\text{  in the matrix }\left(
\begin{array}{cc}
9-\lambda  & -5 \\
 1 & -\lambda -4 \\
\end{array}
\right)$ and solve the system:

$\left(
\begin{array}{cc}
\frac{13}{2}+\frac{\sqrt{149}}{2} & -5 \\
 1 & \frac{\sqrt{149}}{2}-\frac{13}{2} \\
\end{array}
\right).\left(
\begin{array}{c}
v_1 \\
 v_2 \\
\end{array}
\right)=\left(
\begin{array}{c}
0 \\
 0 \\
\end{array}
\right)$

25. In augmented matrix form, the system is written as:

$\left(
\begin{array}{ccc}
\frac{13}{2}+\frac{\sqrt{149}}{2} & -5 & 0 \\
 1 & \frac{\sqrt{149}}{2}-\frac{13}{2} & 0 \\
\end{array}
\right)$

26. Swap row $1\text{ with row }2:$

$\left(
\begin{array}{ccc}
1 & \frac{\sqrt{149}}{2}-\frac{13}{2} & 0 \\
 \frac{13}{2}+\frac{\sqrt{149}}{2} & -5 & 0 \\
\end{array}
\right)$

27. Subtract $\left(\frac{13}{2}+\frac{\sqrt{149}}{2}\right)\, \times \,\text{ (row }1)\text{ from row }2:$

$\left(
\begin{array}{ccc}
1 & \frac{\sqrt{149}}{2}-\frac{13}{2} & 0 \\
 0 & 0 & 0 \\
\end{array}
\right)$

28. Translated back to a matrix equation, the reduced system $\left(
\begin{array}{ccc}
1 & \frac{\sqrt{149}}{2}-\frac{13}{2} & 0 \\
 0 & 0 & 0 \\
\end{array}
\right)$ is:

$\left(
\begin{array}{cc}
1 & \frac{\sqrt{149}}{2}-\frac{13}{2} \\
 0 & 0 \\
\end{array}
\right)\left(
\begin{array}{c}
v_1 \\
 v_2 \\
\end{array}
\right)=\left(
\begin{array}{c}
0 \\
 0 \\
\end{array}
\right)$

29. As a scalar equation, the system $\left(
\begin{array}{cc}
1 & \frac{\sqrt{149}}{2}-\frac{13}{2} \\
 0 & 0 \\
\end{array}
\right)\left(
\begin{array}{c}
v_1 \\
 v_2 \\
\end{array}
\right)=\left(
\begin{array}{c}
0 \\
 0 \\
\end{array}
\right)$ translates to:

$v_1+\left(\frac{\sqrt{149}}{2}-\frac{13}{2}\right) v_2=0$

30. Rewrite the equation as:

$v_1=-\left(\frac{\sqrt{149}}{2}-\frac{13}{2}\right) v_2$

31. According to the above equation:

$v\text{ = }\left(
\begin{array}{c}
v_1 \\
 v_2 \\
\end{array}
\right)\text{ = }\left(
\begin{array}{c}
-\left(\frac{\sqrt{149}}{2}-\frac{13}{2}\right) v_2 \\
 v_2 \\
\end{array}
\right)$

32. Letting $v_2=1\text{ in }\left(
\begin{array}{c}
-\left(\frac{\sqrt{149}}{2}-\frac{13}{2}\right) v_2 \\
 v_2 \\
\end{array}
\right),\text{ we find that }\left(
\begin{array}{c}
\frac{13}{2}-\frac{\sqrt{149}}{2} \\
 1 \\
\end{array}
\right)\text{ is an eigenvector of the matrix }\left(
\begin{array}{cc}
9 & -5 \\
 1 & -4 \\
\end{array}
\right)\text{ associated with the eigenvalue }\frac{5}{2}-\frac{\sqrt{149}}{2}$:

$v=\left(
\begin{array}{c}
\frac{13}{2}-\frac{\sqrt{149}}{2} \\
 1 \\
\end{array}
\right)$

33. We found the following eigenvalue/eigenvector pair:

Answer:

$\begin{array}{c|c}
\text{Eigenvalue }&\text{ Eigenvector }\\
\hline
 \frac{5}{2}+\frac{\sqrt{149}}{2} & \left(
\begin{array}{c}
\frac{13}{2}+\frac{\sqrt{149}}{2} \\
 1 \\
\end{array}
\right) \\
 \frac{5}{2}-\frac{\sqrt{149}}{2} & \left(
\begin{array}{c}
\frac{13}{2}-\frac{\sqrt{149}}{2} \\
 1 \\
\end{array}
\right) \\
\end{array}$

\pagebreak

Find the norm of the vector $\, (8,9,-8,8,-3,-9,5,-10)\,$ :

$\| \, (8,9,-8,8,-3,-9,5,-10)\, \|$

\hrule

1. The formula for the 8-dimensional Euclidean norm comes from the 8-dimensional Pythagorean theorem:

$\left\| \, \left(v_1,v_2,v_3,v_4,v_5,v_6,v_7,v_8\right)\, \right\| =\sqrt{v_1^2+v_2^2+v_3^2+v_4^2+v_5^2+v_6^2+v_7^2+v_8^2}$

2. Substitute $\, (8,9,-8,8,-3,-9,5,-10)\,$ into the formula:

Answer: $\| \, (8,9,-8,8,-3,-9,5,-10)\, \| =\sqrt{8^2+9^2+(-8)^2+8^2+(-3)^2+(-9)^2+5^2+(-10)^2}=2 \sqrt{122}$

\pagebreak

Simplify the following:

$\frac{-1}{16}\left(
\begin{array}{cc}
-6 & -2 \\
 0 & 6 \\
 -8 & 6 \\
\end{array}
\right)$

\hrule

1. $\frac{-1}{16}\left(
\begin{array}{cc}
-6 & -2 \\
 0 & 6 \\
 -8 & 6 \\
\end{array}
\right)=\frac{-1\left(
\begin{array}{cc}
-6 & -2 \\
 0 & 6 \\
 -8 & 6 \\
\end{array}
\right)}{16}$:

$\frac{-1\left(
\begin{array}{cc}
-6 & -2 \\
 0 & 6 \\
 -8 & 6 \\
\end{array}
\right)}{16}$

2. $-1\left(
\begin{array}{cc}
-6 & -2 \\
 0 & 6 \\
 -8 & 6 \\
\end{array}
\right)=\left(
\begin{array}{cc}
-(-6) & -(-2) \\
 -0 & -6 \\
 -(-8) & -6 \\
\end{array}
\right):$

$\frac{\left(
\begin{array}{cc}
-(-6) & -(-2) \\
 0 & -6 \\
 -(-8) & -6 \\
\end{array}
\right)}{16}$

3. $-(-6)\text{ = }6:$

$\frac{1}{16}\left(
\begin{array}{cc}
6 & -(-2) \\
 0 & -6 \\
 -(-8) & -6 \\
\end{array}
\right)$

4. $-(-2)\text{ = }2:$

$\frac{1}{16}\left(
\begin{array}{cc}
6 & 2 \\
 0 & -6 \\
 -(-8) & -6 \\
\end{array}
\right)$

5. $-(-8)\text{ = }8:$

$\frac{1}{16}\left(
\begin{array}{cc}
6 & 2 \\
 0 & -6 \\
 8 & -6 \\
\end{array}
\right)$

6. $\frac{1}{16}\left(
\begin{array}{cc}
6 & 2 \\
 0 & -6 \\
 8 & -6 \\
\end{array}
\right)=\left(
\begin{array}{cc}
\frac{6}{16} & \frac{2}{16} \\
 \frac{0}{16} & \frac{-6}{16} \\
 \frac{8}{16} & \frac{-6}{16} \\
\end{array}
\right):$

$\left(
\begin{array}{cc}
\frac{6}{16} & \frac{2}{16} \\
 \frac{0}{16} & \frac{-6}{16} \\
 \frac{8}{16} & \frac{-6}{16} \\
\end{array}
\right)$

7. The gcd of $6\text{ and 16 is }2,\text{ so }\frac{6}{16}=\frac{2\times 3}{2\times 8}=\frac{2}{2}\times \frac{3}{8}=\frac{3}{8}$:

$\left(
\begin{array}{cc}
\frac{3}{8} & \frac{2}{16} \\
 \frac{0}{16} & \frac{-6}{16} \\
 \frac{8}{16} & \frac{-6}{16} \\
\end{array}
\right)$

8. The gcd of $2\text{ and 16 is }2,\text{ so }\frac{2}{16}=\frac{2\times 1}{2\times 8}=\frac{2}{2}\times \frac{1}{8}=\frac{1}{8}$:

$\left(
\begin{array}{cc}
\frac{3}{8} & \frac{1}{8} \\
 \frac{0}{16} & \frac{-6}{16} \\
 \frac{8}{16} & \frac{-6}{16} \\
\end{array}
\right)$

9. $\frac{0}{16}=0:$

$\left(
\begin{array}{cc}
\frac{3}{8} & \frac{1}{8} \\
 0 & \frac{-6}{16} \\
 \frac{8}{16} & \frac{-6}{16} \\
\end{array}
\right)$

10. The gcd of $-6\text{ and 16 is }2,\text{ so }\frac{-6}{16}=\frac{2 (-3)}{2\times 8}=\frac{2}{2}\times \frac{-3}{8}=\frac{-3}{8}$:

$\left(
\begin{array}{cc}
\frac{3}{8} & \frac{1}{8} \\
 0 & \frac{-3}{8} \\
 \frac{8}{16} & \frac{-6}{16} \\
\end{array}
\right)$

11. The gcd of $8\text{ and 16 is }8,\text{ so }\frac{8}{16}=\frac{8\times 1}{8\times 2}=\frac{8}{8}\times \frac{1}{2}=\frac{1}{2}$:

$\left(
\begin{array}{cc}
\frac{3}{8} & \frac{1}{8} \\
 0 & \frac{-3}{8} \\
 \frac{1}{2} & \frac{-6}{16} \\
\end{array}
\right)$

12. The gcd of $-6\text{ and 16 is }2,\text{ so }\frac{-6}{16}=\frac{2 (-3)}{2\times 8}=\frac{2}{2}\times \frac{-3}{8}=\frac{-3}{8}$:

Answer: $\left(
\begin{array}{cc}
\frac{3}{8} & \frac{1}{8} \\
 0 & \frac{-3}{8} \\
 \frac{1}{2} & \frac{-3}{8} \\
\end{array}
\right)$

\pagebreak

Multiply the following matrices:

$\left(
        \begin{array}{ccc}
2           & 1 & \frac{4}{3}  \\
                \frac{2}{3} & 0 & \frac{1}{3}  \\
                \frac{1}{3} & 3 & -\frac{8}{3} \\
            \end{array}
        \right).\left(
        \begin{array}{cc}
3            & \frac{2}{3} \\
                \frac{2}{3}  & \frac{7}{3} \\
                -\frac{7}{3} & \frac{7}{3} \\
            \end{array}
        \right)$

\hrule

1. $\begin{array}{l}
\text{The dimensions of the first matrix are }3\times 3\text{ and the dimensions of the second matrix are }3\times 2. \\
\text{            This means the dimensions of the product are }3\times 2:                                                                                                  \\
        \end{array}$

$\left(
        \begin{array}{ccc}
2           & 1 & \frac{4}{3}  \\
                \frac{2}{3} & 0 & \frac{1}{3}  \\
                \frac{1}{3} & 3 & -\frac{8}{3} \\
            \end{array}
        \right).\left(
        \begin{array}{cc}
3            & \frac{2}{3} \\
                \frac{2}{3}  & \frac{7}{3} \\
                -\frac{7}{3} & \frac{7}{3} \\
            \end{array}
        \right)=\left(
        \begin{array}{cc}
\_ & \_ \\
                \_ & \_ \\
                \_ & \_ \\
            \end{array}
        \right)$

2. Highlight the $1^{\text{st}}\text{ row and the }1^{\text{st}}$ column:

$\left(
        \begin{array}{ccc}
2           & 1 & \frac{4}{3}  \\
                \frac{2}{3} & 0 & \frac{1}{3}  \\
                \frac{1}{3} & 3 & -\frac{8}{3} \\
            \end{array}
        \right).\left(
        \begin{array}{cc}
3            & \frac{2}{3} \\
                \frac{2}{3}  & \frac{7}{3} \\
                -\frac{7}{3} & \frac{7}{3} \\
            \end{array}
        \right)=\left(
        \begin{array}{cc}
\_ & \_ \\
                \_ & \_ \\
                \_ & \_ \\
            \end{array}
        \right)$

3. $\begin{array}{l}
\text{Multiply corresponding components and add: }2\ 3+\frac{2}{3}+\frac{4 (-7)}{3\ 3}=\frac{32}{9}.                                   \\
\text{            Place this number into the }1^{\text{st}}\text{ row and }1^{\text{st}}\text{ column of the product}: \\
        \end{array}$

$\left(
        \begin{array}{ccc}
2           & 1 & \frac{4}{3}  \\
                \frac{2}{3} & 0 & \frac{1}{3}  \\
                \frac{1}{3} & 3 & -\frac{8}{3} \\
            \end{array}
        \right).\left(
        \begin{array}{cc}
3            & \frac{2}{3} \\
                \frac{2}{3}  & \frac{7}{3} \\
                -\frac{7}{3} & \frac{7}{3} \\
            \end{array}
        \right)=\left(
        \begin{array}{cc}
\frac{32}{9} & \_ \\
                \_                    & \_ \\
                \_                    & \_ \\
            \end{array}
        \right)$

4. Highlight the $1^{\text{st}}\text{ row and the }2^{\text{nd}}$ column:

$\left(
        \begin{array}{ccc}
2           & 1 & \frac{4}{3}  \\
                \frac{2}{3} & 0 & \frac{1}{3}  \\
                \frac{1}{3} & 3 & -\frac{8}{3} \\
            \end{array}
        \right).\left(
        \begin{array}{cc}
3            & \frac{2}{3} \\
                \frac{2}{3}  & \frac{7}{3} \\
                -\frac{7}{3} & \frac{7}{3} \\
            \end{array}
        \right)=\left(
        \begin{array}{cc}
\frac{32}{9} & \_ \\
                \_           & \_ \\
                \_           & \_ \\
            \end{array}
        \right)$

5. $\begin{array}{l}
\text{Multiply corresponding components and add: }\frac{2\ 2}{3}+\frac{7}{3}+\frac{4\ 7}{3\ 3}=\frac{61}{9}.                           \\
\text{            Place this number into the }1^{\text{st}}\text{ row and }2^{\text{nd}}\text{ column of the product}: \\
        \end{array}$

$\left(
        \begin{array}{ccc}
2           & 1 & \frac{4}{3}  \\
                \frac{2}{3} & 0 & \frac{1}{3}  \\
                \frac{1}{3} & 3 & -\frac{8}{3} \\
            \end{array}
        \right).\left(
        \begin{array}{cc}
3            & \frac{2}{3} \\
                \frac{2}{3}  & \frac{7}{3} \\
                -\frac{7}{3} & \frac{7}{3} \\
            \end{array}
        \right)=\left(
        \begin{array}{cc}
\frac{32}{9} & \frac{61}{9} \\
                \_           & \_                    \\
                \_           & \_                    \\
            \end{array}
        \right)$

6. Highlight the $2^{\text{nd}}\text{ row and the }1^{\text{st}}$ column:

$\left(
        \begin{array}{ccc}
2           & 1 & \frac{4}{3}  \\
                \frac{2}{3} & 0 & \frac{1}{3}  \\
                \frac{1}{3} & 3 & -\frac{8}{3} \\
            \end{array}
        \right).\left(
        \begin{array}{cc}
3            & \frac{2}{3} \\
                \frac{2}{3}  & \frac{7}{3} \\
                -\frac{7}{3} & \frac{7}{3} \\
            \end{array}
        \right)=\left(
        \begin{array}{cc}
\frac{32}{9} & \frac{61}{9} \\
                \_           & \_           \\
                \_           & \_           \\
            \end{array}
        \right)$

7. $\begin{array}{l}
\text{Multiply corresponding components and add: }\frac{2\ 3}{3}+\frac{0\ 2}{3}-\frac{7}{3\ 3}=\frac{11}{9}.                           \\
\text{            Place this number into the }2^{\text{nd}}\text{ row and }1^{\text{st}}\text{ column of the product}: \\
        \end{array}$

$\left(
        \begin{array}{ccc}
2           & 1 & \frac{4}{3}  \\
                \frac{2}{3} & 0 & \frac{1}{3}  \\
                \frac{1}{3} & 3 & -\frac{8}{3} \\
            \end{array}
        \right).\left(
        \begin{array}{cc}
3            & \frac{2}{3} \\
                \frac{2}{3}  & \frac{7}{3} \\
                -\frac{7}{3} & \frac{7}{3} \\
            \end{array}
        \right)=\left(
        \begin{array}{cc}
\frac{32}{9}          & \frac{61}{9} \\
                \frac{11}{9} & \_           \\
                \_                    & \_           \\
            \end{array}
        \right)$

8. Highlight the $2^{\text{nd}}\text{ row and the }2^{\text{nd}}$ column:

$\left(
        \begin{array}{ccc}
2           & 1 & \frac{4}{3}  \\
                \frac{2}{3} & 0 & \frac{1}{3}  \\
                \frac{1}{3} & 3 & -\frac{8}{3} \\
            \end{array}
        \right).\left(
        \begin{array}{cc}
3            & \frac{2}{3} \\
                \frac{2}{3}  & \frac{7}{3} \\
                -\frac{7}{3} & \frac{7}{3} \\
            \end{array}
        \right)=\left(
        \begin{array}{cc}
\frac{32}{9} & \frac{61}{9} \\
                \frac{11}{9} & \_           \\
                \_           & \_           \\
            \end{array}
        \right)$

9. $\begin{array}{l}
\text{Multiply corresponding components and add: }\frac{2\ 2}{3\ 3}+\frac{0\ 7}{3}+\frac{7}{3\ 3}=\frac{11}{9}.                        \\
\text{            Place this number into the }2^{\text{nd}}\text{ row and }2^{\text{nd}}\text{ column of the product}: \\
        \end{array}$

$\left(
        \begin{array}{ccc}
2           & 1 & \frac{4}{3}  \\
                \frac{2}{3} & 0 & \frac{1}{3}  \\
                \frac{1}{3} & 3 & -\frac{8}{3} \\
            \end{array}
        \right).\left(
        \begin{array}{cc}
3            & \frac{2}{3} \\
                \frac{2}{3}  & \frac{7}{3} \\
                -\frac{7}{3} & \frac{7}{3} \\
            \end{array}
        \right)=\left(
        \begin{array}{cc}
\frac{32}{9} & \frac{61}{9}          \\
                \frac{11}{9} & \frac{11}{9} \\
                \_           & \_                    \\
            \end{array}
        \right)$

10. Highlight the $3^{\text{rd}}\text{ row and the }1^{\text{st}}$ column:

$\left(
        \begin{array}{ccc}
2           & 1 & \frac{4}{3}  \\
                \frac{2}{3} & 0 & \frac{1}{3}  \\
                \frac{1}{3} & 3 & -\frac{8}{3} \\
            \end{array}
        \right).\left(
        \begin{array}{cc}
3            & \frac{2}{3} \\
                \frac{2}{3}  & \frac{7}{3} \\
                -\frac{7}{3} & \frac{7}{3} \\
            \end{array}
        \right)=\left(
        \begin{array}{cc}
\frac{32}{9} & \frac{61}{9} \\
                \frac{11}{9} & \frac{11}{9} \\
                \_           & \_           \\
            \end{array}
        \right)$

11. $\begin{array}{l}
\text{Multiply corresponding components and add: }\frac{3}{3}+\frac{3\ 2}{3}+\left(-\frac{8}{3}\right)\, \left(-\frac{7}{3}\right)=\frac{83}{9}. \\
\text{            Place this number into the }3^{\text{rd}}\text{ row and }1^{\text{st}}\text{ column of the product}:           \\
        \end{array}$

$\left(
        \begin{array}{ccc}
2           & 1 & \frac{4}{3}  \\
                \frac{2}{3} & 0 & \frac{1}{3}  \\
                \frac{1}{3} & 3 & -\frac{8}{3} \\
            \end{array}
        \right).\left(
        \begin{array}{cc}
3            & \frac{2}{3} \\
                \frac{2}{3}  & \frac{7}{3} \\
                -\frac{7}{3} & \frac{7}{3} \\
            \end{array}
        \right)=\left(
        \begin{array}{cc}
\frac{32}{9}          & \frac{61}{9} \\
                \frac{11}{9}          & \frac{11}{9} \\
                \frac{83}{9} & \_           \\
            \end{array}
        \right)$

12. Highlight the $3^{\text{rd}}\text{ row and the }2^{\text{nd}}$ column:

$\left(
        \begin{array}{ccc}
2           & 1 & \frac{4}{3}  \\
                \frac{2}{3} & 0 & \frac{1}{3}  \\
                \frac{1}{3} & 3 & -\frac{8}{3} \\
            \end{array}
        \right).\left(
        \begin{array}{cc}
3            & \frac{2}{3} \\
                \frac{2}{3}  & \frac{7}{3} \\
                -\frac{7}{3} & \frac{7}{3} \\
            \end{array}
        \right)=\left(
        \begin{array}{cc}
\frac{32}{9} & \frac{61}{9} \\
                \frac{11}{9} & \frac{11}{9} \\
                \frac{83}{9} & \_           \\
            \end{array}
        \right)$

13. $\begin{array}{l}
\text{Multiply corresponding components and add: }\frac{2}{3\ 3}+\frac{3\ 7}{3}+\left(-\frac{8}{3}\right)\, \times \, \frac{7}{3}=1.   \\
\text{            Place this number into the }3^{\text{rd}}\text{ row and }2^{\text{nd}}\text{ column of the product}: \\
        \end{array}$

Answer: $\left(
                    \begin{array}{ccc}
2           & 1 & \frac{4}{3}  \\
                            \frac{2}{3} & 0 & \frac{1}{3}  \\
                            \frac{1}{3} & 3 & -\frac{8}{3} \\
                        \end{array}
                    \right).\left(
                    \begin{array}{cc}
3            & \frac{2}{3} \\
                            \frac{2}{3}  & \frac{7}{3} \\
                            -\frac{7}{3} & \frac{7}{3} \\
                        \end{array}
                    \right)=\left(
                    \begin{array}{cc}
\frac{32}{9} & \frac{61}{9} \\
                            \frac{11}{9} & \frac{11}{9} \\
                            \frac{83}{9} & 1   \\
                        \end{array}
                    \right)$

\pagebreak

Find the null space of the matrix $M:$

$M=\left(
\begin{array}{ccc}
1 & -7 & 7 \\
 -2 & 2 & 8 \\
\end{array}
\right)$

\hrule

1. The null space of matrix $M=\left(
\begin{array}{ccc}
1 & -7 & 7 \\
 -2 & 2 & 8 \\
\end{array}
\right)\text{ is the set of all vectors }v=\left(
\begin{array}{c}
x_1 \\
 x_2 \\
 x_3 \\
\end{array}
\right)\text{ such that }M.v=0:$

$\left(
\begin{array}{ccc}
1 & -7 & 7 \\
 -2 & 2 & 8 \\
\end{array}
\right).\left(
\begin{array}{c}
x_1 \\
 x_2 \\
 x_3 \\
\end{array}
\right)=\left(
\begin{array}{c}
0 \\
 0 \\
\end{array}
\right)$

2. Reduce the matrix $\left(
\begin{array}{ccc}
1 & -7 & 7 \\
 -2 & 2 & 8 \\
\end{array}
\right)$ to row echelon form:

$\left(
\begin{array}{ccc}
1 & -7 & 7 \\
 -2 & 2 & 8 \\
\end{array}
\right)$

3. Add $2\, \times \,\text{ (row }1)\text{ to row }2:$

$\left(
\begin{array}{ccc}
1 & -7 & 7 \\
 0 & -12 & 22 \\
\end{array}
\right)$

4. Divide row $2\text{ by }-12:$

$\left(
\begin{array}{ccc}
1 & -7 & 7 \\
 0 & 1 & -\frac{11}{6} \\
\end{array}
\right)$

5. Add $7\, \times \,\text{ (row }2)\text{ to row }1:$

$\left(
\begin{array}{ccc}
1 & 0 & -\frac{35}{6} \\
 0 & 1 & -\frac{11}{6} \\
\end{array}
\right)$

6. Free variables in the null space $\left(
\begin{array}{c}
x_1 \\
 x_2 \\
 x_3 \\
\end{array}
\right)\text{ correspond to the columns in }\left(
\begin{array}{ccc}
1 & 0 & -\frac{35}{6} \\
 0 & 1 & -\frac{11}{6} \\
\end{array}
\right)$ which have no pivot.

Column $3\text{ is the only column with no pivot, so we may take }x_3$ to be the only free variable

7. Multiply out the reduced matrix $\left(
\begin{array}{ccc}
1 & 0 & -\frac{35}{6} \\
 0 & 1 & -\frac{11}{6} \\
\end{array}
\right)\text{ with the proposed solution vector }\left(
\begin{array}{c}
x_1 \\
 x_2 \\
 x_3 \\
\end{array}
\right):$

$\left(
\begin{array}{ccc}
1 & 0 & -\frac{35}{6} \\
 0 & 1 & -\frac{11}{6} \\
\end{array}
\right).\left(
\begin{array}{c}
x_1 \\
 x_2 \\
 x_3 \\
\end{array}
\right)=\left(
\begin{array}{c}
x_1-\frac{35 x_3}{6} \\
 x_2-\frac{11 x_3}{6} \\
\end{array}
\right)=\left(
\begin{array}{c}
0 \\
 0 \\
\end{array}
\right)$

8. Solve the equations $\{
\begin{array}{l}
x_1-\frac{35 x_3}{6}=0 \\
 x_2-\frac{11 x_3}{6}=0 \\
\end{array}
\text{  for }x_1\text{ and }x_2:$

$\{
\begin{array}{l}
x_1=\frac{35 x_3}{6} \\
 x_2=\frac{11 x_3}{6} \\
\end{array}$

9. Rewrite $v\text{ in terms of the free variable }x_3,\text{ and assign it an arbitrary real value of }x:$

$v=\left(
\begin{array}{c}
x_1 \\
 x_2 \\
 x_3 \\
\end{array}
\right)=\left(
\begin{array}{c}
\frac{35 x_3}{6} \\
 \frac{11 x_3}{6} \\
 x_3 \\
\end{array}
\right)=\left(
\begin{array}{c}
\frac{35 x}{6} \\
 \frac{11 x}{6} \\
 x \\
\end{array}
\right)\text{ for }x\in \mathbb{R}$

10. Since $x\text{ is taken from }\mathbb{R},\text{ we can replace it with }6 x:$

$\left(
\begin{array}{c}
\frac{35 x}{6} \\
 \frac{11 x}{6} \\
 x \\
\end{array}
\right)\, \rightarrow \, \left(
\begin{array}{c}
\frac{35 (6 x)}{6} \\
 \frac{11 (6 x)}{6} \\
 6 x \\
\end{array}
\right)=\left(
\begin{array}{c}
35 x \\
 11 x \\
 6 x \\
\end{array}
\right)\text{ for }x\in \mathbb{R}$

11. Rewrite the solution vector $v=\left(
\begin{array}{c}
35 x \\
 11 x \\
 6 x \\
\end{array}
\right)$ in set notation:

Answer: $\{\, (35 x,11 x,6 x)\,\text{ $\, $: }x\in \mathbb{R}\}$

\pagebreak

Find the inverse:

$\left(
\begin{array}{cc}
-1 & 4 \\
 2 & -5 \\
\end{array}
\right)^{-1}$

\hrule

1. Using a formula for the inverse of a 2$\times $2 matrix, $\left(
\begin{array}{cc}
-1 & 4 \\
 2 & -5 \\
\end{array}
\right)^{-1}=\frac{1}{-(-5)-2\times 4}\left(
\begin{array}{cc}
-5 & -4 \\
 -2 & -1 \\
\end{array}
\right):$

$\frac{1}{-(-5)-2\times 4}\left(
\begin{array}{cc}
-5 & -4 \\
 -2 & -1 \\
\end{array}
\right)$

2. Simplify: $\frac{1}{-(-5)-2\times 4}=-\frac{1}{3}$:

Answer: $-\frac{1}{3}\left(
\begin{array}{cc}
-5 & -4 \\
 -2 & -1 \\
\end{array}
\right)$

\pagebreak

Find the rank of the matrix:

$M=\left(
\begin{array}{c}
-5 \\
\end{array}
\right)$

\hrule

1. As each row of matrix $M$ has only zeroes to the left of the main diagonal, it is in row echelon form:

$\left(
\begin{array}{c}
-5 \\
\end{array}
\right)$

2. There is $1\text{ nonzero row in the row echelon matrix, so the rank is }1:$

Answer: rank$\left(
\begin{array}{c}
-5 \\
\end{array}
\right)=1$

\pagebreak

Do row reduction:

$\left(
\begin{array}{cc}
7 & -3 \\
 -8 & -9 \\
 -9 & 4 \\
\end{array}
\right)$

\hrule

1. Swap row $1\text{ with row }3:$

$\left(
\begin{array}{cc}
-9 & 4 \\
 -8 & -9 \\
 7 & -3 \\
\end{array}
\right)$

2. Subtract $\frac{8}{9}\, \times \,\text{ (row }1)\text{ from row }2:$

$\left(
\begin{array}{cc}
-9 & 4 \\
 0 & -\frac{113}{9} \\
 7 & -3 \\
\end{array}
\right)$

3. Add $\frac{7}{9}\, \times \,\text{ (row }1)\text{ to row }3:$

$\left(
\begin{array}{cc}
-9 & 4 \\
 0 & -\frac{113}{9} \\
 0 & \frac{1}{9} \\
\end{array}
\right)$

4. Add $\frac{1}{113}\, \times \,\text{ (row }2)\text{ to row }3:$

$\left(
\begin{array}{cc}
-9 & 4 \\
 0 & -\frac{113}{9} \\
 0 & 0 \\
\end{array}
\right)$

5. Multiply row $2\text{ by }-\frac{9}{113}$:

$\left(
\begin{array}{cc}
-9 & 4 \\
 0 & 1 \\
 0 & 0 \\
\end{array}
\right)$

6. Subtract $4\, \times \,\text{ (row }2)\text{ from row }1:$

$\left(
\begin{array}{cc}
-9 & 0 \\
 0 & 1 \\
 0 & 0 \\
\end{array}
\right)$

7. Divide row $1\text{ by }-9:$

$\left(
\begin{array}{cc}
1 & 0 \\
 0 & 1 \\
 0 & 0 \\
\end{array}
\right)$

8. This matrix is now in reduced row echelon form.$\backslash $nAll non-zero rows are above rows of all zeros:

$\left(
\begin{array}{cc}
1 & 0 \\
 0 & 1 \\
 0 & 0 \\
\end{array}
\right)$

9. Each pivot is 1 and is strictly to the right of every pivot above it:

$\left(
\begin{array}{cc}
1 & 0 \\
 0 & 1 \\
 0 & 0 \\
\end{array}
\right)$

10. Each pivot is the only non-zero entry in its column:

Answer: $\left(
\begin{array}{cc}
1 & 0 \\
 0 & 1 \\
 0 & 0 \\
\end{array}
\right)$

\pagebreak

Simplify the following:

$\left(
\begin{array}{ccc}
\frac{7}{2} & \frac{28}{3} & \frac{31}{6} \\
 -\frac{14}{3} & -2 & -\frac{5}{3} \\
\end{array}
\right)-\left(
\begin{array}{ccc}
\frac{13}{3} & \frac{15}{2} & \frac{55}{6} \\
 \frac{2}{3} & -\frac{31}{6} & \frac{19}{2} \\
\end{array}
\right)$

\hrule

1. $-1\left(
\begin{array}{ccc}
\frac{13}{3} & \frac{15}{2} & \frac{55}{6} \\
 \frac{2}{3} & -\frac{31}{6} & \frac{19}{2} \\
\end{array}
\right)=\left(
\begin{array}{ccc}
-\frac{13}{3} & -\frac{15}{2} & -\frac{55}{6} \\
 -\frac{2}{3} & -\frac{-31}{6} & -\frac{19}{2} \\
\end{array}
\right):$

$\left(
\begin{array}{ccc}
\frac{7}{2} & \frac{28}{3} & \frac{31}{6} \\
 -\frac{14}{3} & -2 & -\frac{5}{3} \\
\end{array}
\right)+\left(
\begin{array}{ccc}
-\frac{13}{3} & -\frac{15}{2} & -\frac{55}{6} \\
 -\frac{2}{3} & -\frac{-31}{6} & -\frac{19}{2} \\
\end{array}
\right)$

2. $-(-31)\text{ = }31:$

$\left(
\begin{array}{ccc}
\frac{7}{2} & \frac{28}{3} & \frac{31}{6} \\
 -\frac{14}{3} & -2 & -\frac{5}{3} \\
\end{array}
\right)+\left(
\begin{array}{ccc}
-\frac{13}{3} & -\frac{15}{2} & -\frac{55}{6} \\
 -\frac{2}{3} & \frac{31}{6} & -\frac{19}{2} \\
\end{array}
\right)$

3. $\left(
\begin{array}{ccc}
\frac{7}{2} & \frac{28}{3} & \frac{31}{6} \\
 -\frac{14}{3} & -2 & -\frac{5}{3} \\
\end{array}
\right)+\left(
\begin{array}{ccc}
-\frac{13}{3} & -\frac{15}{2} & -\frac{55}{6} \\
 -\frac{2}{3} & \frac{31}{6} & -\frac{19}{2} \\
\end{array}
\right)=\left(
\begin{array}{ccc}
\frac{7}{2}-\frac{13}{3} & \frac{28}{3}-\frac{15}{2} & \frac{31}{6}-\frac{55}{6} \\
 -\frac{14}{3}-\frac{2}{3} & -2+\frac{31}{6} & -\frac{5}{3}-\frac{19}{2} \\
\end{array}
\right):$

$\left(
\begin{array}{ccc}
\frac{7}{2}-\frac{13}{3} & \frac{28}{3}-\frac{15}{2} & \frac{31}{6}-\frac{55}{6} \\
 -\frac{14}{3}-\frac{2}{3} & -2+\frac{31}{6} & -\frac{5}{3}-\frac{19}{2} \\
\end{array}
\right)$

4. Put $\frac{7}{2}-\frac{13}{3}\text{ over the common denominator }6. \frac{7}{2}-\frac{13}{3}\text{ = }\frac{3\times 7}{6}+\frac{2 (-13)}{6}$:

$\left(
\begin{array}{ccc}
\frac{3\times 7}{6}+\frac{2 (-13)}{6} & \frac{28}{3}-\frac{15}{2} & \frac{31}{6}-\frac{55}{6} \\
 -\frac{14}{3}-\frac{2}{3} & -2+\frac{31}{6} & -\frac{5}{3}-\frac{19}{2} \\
\end{array}
\right)$

5. $3\times 7\text{ = }21:$

$\left(
\begin{array}{ccc}
\frac{21}{6}+\frac{2 (-13)}{6} & \frac{28}{3}-\frac{15}{2} & \frac{31}{6}-\frac{55}{6} \\
 -\frac{14}{3}-\frac{2}{3} & -2+\frac{31}{6} & -\frac{5}{3}-\frac{19}{2} \\
\end{array}
\right)$

6. $2 (-13)\text{ = }-26:$

$\left(
\begin{array}{ccc}
\frac{21}{6}+\frac{-26}{6} & \frac{28}{3}-\frac{15}{2} & \frac{31}{6}-\frac{55}{6} \\
 -\frac{14}{3}-\frac{2}{3} & -2+\frac{31}{6} & -\frac{5}{3}-\frac{19}{2} \\
\end{array}
\right)$

7. $\frac{21}{6}-\frac{26}{6}\text{ = }\frac{21-26}{6}$:

$\left(
\begin{array}{ccc}
\frac{21-26}{6} & \frac{28}{3}-\frac{15}{2} & \frac{31}{6}-\frac{55}{6} \\
 -\frac{14}{3}-\frac{2}{3} & -2+\frac{31}{6} & -\frac{5}{3}-\frac{19}{2} \\
\end{array}
\right)$

8. $21-26\text{ = }-(26-21):$

$\left(
\begin{array}{ccc}
\frac{-(26-21)}{6} & \frac{28}{3}-\frac{15}{2} & \frac{31}{6}-\frac{55}{6} \\
 -\frac{14}{3}-\frac{2}{3} & -2+\frac{31}{6} & -\frac{5}{3}-\frac{19}{2} \\
\end{array}
\right)$

9. $\begin{array}{c}
\begin{array}{ccc}
\text{ }& 2 & 6 \\
 - & 2 & 1 \\
\hline
\text{  }& 0 & 5 \\
\end{array}
 \\
\end{array}$

:

$\left(
\begin{array}{ccc}
\frac{-5}{6} & \frac{28}{3}-\frac{15}{2} & \frac{31}{6}-\frac{55}{6} \\
 -\frac{14}{3}-\frac{2}{3} & -2+\frac{31}{6} & -\frac{5}{3}-\frac{19}{2} \\
\end{array}
\right)$

10. Put $\frac{28}{3}-\frac{15}{2}\text{ over the common denominator }6. \frac{28}{3}-\frac{15}{2}\text{ = }\frac{2\times 28}{6}+\frac{3 (-15)}{6}$:

$\left(
\begin{array}{ccc}
-\frac{5}{6} & \frac{2\times 28}{6}+\frac{3 (-15)}{6} & \frac{31}{6}-\frac{55}{6} \\
 -\frac{14}{3}-\frac{2}{3} & -2+\frac{31}{6} & -\frac{5}{3}-\frac{19}{2} \\
\end{array}
\right)$

11. $2\times 28\text{ = }56:$

$\left(
\begin{array}{ccc}
-\frac{5}{6} & \frac{56}{6}+\frac{3 (-15)}{6} & \frac{31}{6}-\frac{55}{6} \\
 -\frac{14}{3}-\frac{2}{3} & -2+\frac{31}{6} & -\frac{5}{3}-\frac{19}{2} \\
\end{array}
\right)$

12. $3 (-15)\text{ = }-45:$

$\left(
\begin{array}{ccc}
-\frac{5}{6} & \frac{56}{6}+\frac{-45}{6} & \frac{31}{6}-\frac{55}{6} \\
 -\frac{14}{3}-\frac{2}{3} & -2+\frac{31}{6} & -\frac{5}{3}-\frac{19}{2} \\
\end{array}
\right)$

13. $\frac{56}{6}-\frac{45}{6}\text{ = }\frac{56-45}{6}$:

$\left(
\begin{array}{ccc}
-\frac{5}{6} & \frac{56-45}{6} & \frac{31}{6}-\frac{55}{6} \\
 -\frac{14}{3}-\frac{2}{3} & -2+\frac{31}{6} & -\frac{5}{3}-\frac{19}{2} \\
\end{array}
\right)$

14. $\begin{array}{c}
\begin{array}{ccc}
\text{ }& 5 & 6 \\
 - & 4 & 5 \\
\hline
\text{  }& 1 & 1 \\
\end{array}
 \\
\end{array}$

:

$\left(
\begin{array}{ccc}
-\frac{5}{6} & \frac{11}{6} & \frac{31}{6}-\frac{55}{6} \\
 -\frac{14}{3}-\frac{2}{3} & -2+\frac{31}{6} & -\frac{5}{3}-\frac{19}{2} \\
\end{array}
\right)$

15. $\frac{31}{6}-\frac{55}{6}\text{ = }\frac{31-55}{6}$:

$\left(
\begin{array}{ccc}
-\frac{5}{6} & \frac{11}{6} & \frac{31-55}{6} \\
 -\frac{14}{3}-\frac{2}{3} & -2+\frac{31}{6} & -\frac{5}{3}-\frac{19}{2} \\
\end{array}
\right)$

16. $31-55\text{ = }-(55-31):$

$\left(
\begin{array}{ccc}
-\frac{5}{6} & \frac{11}{6} & \frac{-(55-31)}{6} \\
 -\frac{14}{3}-\frac{2}{3} & -2+\frac{31}{6} & -\frac{5}{3}-\frac{19}{2} \\
\end{array}
\right)$

17. $\begin{array}{c}
\begin{array}{ccc}
\text{ }& 5 & 5 \\
 - & 3 & 1 \\
\hline
\text{  }& 2 & 4 \\
\end{array}
 \\
\end{array}$

:

$\left(
\begin{array}{ccc}
-\frac{5}{6} & \frac{11}{6} & \frac{-24}{6} \\
 -\frac{14}{3}-\frac{2}{3} & -2+\frac{31}{6} & -\frac{5}{3}-\frac{19}{2} \\
\end{array}
\right)$

18. $\frac{24}{6}=\frac{6\times 4}{6}=4:$

$\left(
\begin{array}{ccc}
-\frac{5}{6} & \frac{11}{6} & -4 \\
 -\frac{14}{3}-\frac{2}{3} & -2+\frac{31}{6} & -\frac{5}{3}-\frac{19}{2} \\
\end{array}
\right)$

19. $-\frac{14}{3}-\frac{2}{3}\text{ = }\frac{-14-2}{3}$:

$\left(
\begin{array}{ccc}
-\frac{5}{6} & \frac{11}{6} & -4 \\
 \frac{-14-2}{3} & -2+\frac{31}{6} & -\frac{5}{3}-\frac{19}{2} \\
\end{array}
\right)$

20. $-14-2=-(14+2):$

$\left(
\begin{array}{ccc}
-\frac{5}{6} & \frac{11}{6} & -4 \\
 \frac{-(14+2)}{3} & -2+\frac{31}{6} & -\frac{5}{3}-\frac{19}{2} \\
\end{array}
\right)$

21. $14+2=16:$

$\left(
\begin{array}{ccc}
-\frac{5}{6} & \frac{11}{6} & -4 \\
 \frac{-16}{3} & -2+\frac{31}{6} & -\frac{5}{3}-\frac{19}{2} \\
\end{array}
\right)$

22. Put $-2+\frac{31}{6}\text{ over the common denominator }6. -2+\frac{31}{6}\text{ = }\frac{6 (-2)}{6}+\frac{31}{6}$:

$\left(
\begin{array}{ccc}
-\frac{5}{6} & \frac{11}{6} & -4 \\
 -\frac{16}{3} & \frac{6 (-2)}{6}+\frac{31}{6} & -\frac{5}{3}-\frac{19}{2} \\
\end{array}
\right)$

23. $6 (-2)\text{ = }-12:$

$\left(
\begin{array}{ccc}
-\frac{5}{6} & \frac{11}{6} & -4 \\
 -\frac{16}{3} & \frac{-12}{6}+\frac{31}{6} & -\frac{5}{3}-\frac{19}{2} \\
\end{array}
\right)$

24. $-\frac{12}{6}+\frac{31}{6}\text{ = }\frac{-12+31}{6}$:

$\left(
\begin{array}{ccc}
-\frac{5}{6} & \frac{11}{6} & -4 \\
 -\frac{16}{3} & \frac{-12+31}{6} & -\frac{5}{3}-\frac{19}{2} \\
\end{array}
\right)$

25. $-12+31=19:$

$\left(
\begin{array}{ccc}
-\frac{5}{6} & \frac{11}{6} & -4 \\
 -\frac{16}{3} & \frac{19}{6} & -\frac{5}{3}-\frac{19}{2} \\
\end{array}
\right)$

26. Put $-\frac{5}{3}-\frac{19}{2}\text{ over the common denominator }6. -\frac{5}{3}-\frac{19}{2}\text{ = }\frac{2 (-5)}{6}+\frac{3 (-19)}{6}$:

$\left(
\begin{array}{ccc}
-\frac{5}{6} & \frac{11}{6} & -4 \\
 -\frac{16}{3} & \frac{19}{6} & \frac{2 (-5)}{6}+\frac{3 (-19)}{6} \\
\end{array}
\right)$

27. $2 (-5)\text{ = }-10:$

$\left(
\begin{array}{ccc}
-\frac{5}{6} & \frac{11}{6} & -4 \\
 -\frac{16}{3} & \frac{19}{6} & \frac{-10}{6}+\frac{3 (-19)}{6} \\
\end{array}
\right)$

28. $3 (-19)\text{ = }-57:$

$\left(
\begin{array}{ccc}
-\frac{5}{6} & \frac{11}{6} & -4 \\
 -\frac{16}{3} & \frac{19}{6} & \frac{-10}{6}+\frac{-57}{6} \\
\end{array}
\right)$

29. $-\frac{10}{6}-\frac{57}{6}\text{ = }\frac{-10-57}{6}$:

$\left(
\begin{array}{ccc}
-\frac{5}{6} & \frac{11}{6} & -4 \\
 -\frac{16}{3} & \frac{19}{6} & \frac{-10-57}{6} \\
\end{array}
\right)$

30. $-10-57=-(10+57):$

$\left(
\begin{array}{ccc}
-\frac{5}{6} & \frac{11}{6} & -4 \\
 -\frac{16}{3} & \frac{19}{6} & \frac{-(10+57)}{6} \\
\end{array}
\right)$

31. $\begin{array}{c}
\begin{array}{ccc}
\hline
\text{  }& 5 & 7 \\
\hline
 + & 1 & 0 \\
\text{  }& 6 & 7 \\
\end{array}
 \\
\end{array}$

:

Answer: $\left(
\begin{array}{ccc}
-\frac{5}{6} & \frac{11}{6} & -4 \\
 -\frac{16}{3} & \frac{19}{6} & \frac{-67}{6} \\
\end{array}
\right)$

\pagebreak

Find the trace of the matrix:

$\left(
\begin{array}{cccc}
-3 e & -4 e & -2 e & -2 e \\
 -2 e & -2 e & -4 e & -4 e \\
 e & 3 e & 2 e & -3 e \\
 -2 e & 3 e & -3 e & 3 e \\
\end{array}
\right)$

\hrule

1. Locate the elements on the main diagonal:

$\left(
\begin{array}{cccc}
-3 e & -4 e & -2 e & -2 e \\
 -2 e & -2 e & -4 e & -4 e \\
 e & 3 e & 2 e & -3 e \\
 -2 e & 3 e & -3 e & 3 e \\
\end{array}
\right)$

2. Write out the sum of the elements on the main diagonal:

$-3 e\,\text{ +$\, $}-2 e\,\text{ +$\, $}2 e\,\text{ +$\, $}3 e$

3. Add like terms. $-3 e-2 e+2 e+3 e\text{ = }0:$

Answer: $0$

\pagebreak

Convert the following to base $26:$

$6876_{10}$

\hrule

1. Determine the powers of 26 that will be used as the places of the digits in the base-26 representation of $6876:$

$\begin{array}{|c|c|c}
\hline
\text{ Power }&\text{ Base}^{\text{Power}} &\text{ Place value }\\
\hline
 3 & 26^3 & 17576 \\
\hline
 2 & 26^2 & 676 \\
 1 & 26^1 & 26 \\
 0 & 26^0 & 1 \\
\end{array}$

2. Label each place of the base-26 representation of 6876 with the appropriate power of $26:$

$\begin{array}{ccccccc}
\text{Place }&   &   & 26^2 & 26^1 & 26^0 &   \\
   &   &   & \downarrow  & \downarrow  & \downarrow  &   \\
 6876_{10} & = & ( & \_\_ & \_\_ & \_\_ & )_{\text{}_{26}} \\
\end{array}$

3. Determine the value of the first digit from the right of 6876 in base $26:$

$\begin{array}{l}
\frac{6876}{26}=264\text{ with remainder }12 \\
 
\begin{array}{ccccccc}
\text{Place }&   &   & 26^2 & 26^1 & 26^0 &   \\
   &   &   & \downarrow  & \downarrow  & \downarrow  &   \\
 6876_{10} & = & ( & \_\_ & \_\_ & 12 & )_{\text{}_{26}} \\
\end{array}
 \\
\end{array}$

4. Determine the value of the next digit from the right of 6876 in base $26:$

$\begin{array}{l}
\frac{264}{26}=10\text{ with remainder }4 \\
 
\begin{array}{ccccccc}
\text{Place }&   &   & 26^2 & 26^1 & 26^0 &   \\
   &   &   & \downarrow  & \downarrow  & \downarrow  &   \\
 6876_{10} & = & ( & \_\_ & 4 & 12 & )_{\text{}_{26}} \\
\end{array}
 \\
\end{array}$

5. Determine the value of the last remaining digit of 6876 in base $26:$

$\begin{array}{l}
\frac{10}{26}=0\text{ with remainder }10 \\
 
\begin{array}{ccccccc}
\text{Place }&   &   & 26^2 & 26^1 & 26^0 &   \\
   &   &   & \downarrow  & \downarrow  & \downarrow  &   \\
 6876_{10} & = & ( & 10 & 4 & 12 & )_{\text{}_{26}} \\
\end{array}
 \\
\end{array}$

6. Convert all digits that are greater than 9 to their base 26 alphabetical equivalent:

$\begin{array}{ccc}
\text{Base-26 digit value }&   &\text{ Base-26 digit }\\
 10 & \rightarrow  &\text{ a }\\
 4 & \rightarrow  & 4 \\
 12 & \rightarrow  &\text{ c }\\
\end{array}$

7. The number $6876_{10}\text{ is equivalent to a4c}_{26}\text{ in base }26.$

Answer: $6876_{10}=\text{a4c}_{26}$

\pagebreak

Factor the following integer:

$2$

\hrule

1. No primes less than $2\text{ divide into it. Therefore }2$ is prime:

$2=2$

2. Since $2\text{ is prime it has an exponent of }1:$

Answer: $2=2^1$

\pagebreak

Find the greatest common divisor:

$\gcd (-783,-198)$

\hrule

1. Any divisor of a negative number is also a divisor of its absolute value:

$\gcd (-783,-198)=\gcd (783,198)$

2. $\begin{array}{l}
\text{Find the divisors of each integer and select the largest element they have in common}: \\
\text{ The divisors of 783 are}: \\
\end{array}$

$1,3,9,27,29,87,261,783$

3. The divisors of 198 are:

$1,2,3,6,9,11,18,22,33,66,99,198$

4. $\begin{array}{l}
\text{The largest number common to both divisor lists is }9: \\
\text{ divisors of 783: }1,3,9,27,29,87,261,783 \\
\text{ divisors of 198: }1,2,3,6,9,11,18,22,33,66,99,198 \\
\end{array}$

Answer: $\gcd (-783,-198)=9$

\pagebreak

Test for primality:

is $0$ a prime number?

\hrule

1. Since every positive integer is a divisor of $0,$ it is not prime:

Answer: $0$ is not a prime number

\pagebreak

Find the least common multiple:

lcm$(-53,-39)$

\hrule

1. Any multiple of a negative number is also a multiple of its absolute value:

lcm$(-53,-39)=\text{lcm}(53,39)$

2. $\begin{array}{l}
\text{Find the prime factorization of each integer}: \\
\text{ The prime factorization of 53 is}: \\
\end{array}$

$53=53^1$

3. The prime factorization of 39 is:

$39=3\times 13$

4. $\begin{array}{l}
\text{Find the largest power of each prime factor }\\
\text{ The largest power of }3\text{ that appears in the prime factorizations is }3^1 \\
\text{ The largest power of 13 that appears in the prime factorizations is }13^1 \\
\text{ The largest power of 53 that appears in the prime factorizations is }53^1 \\
\text{ Therefore lcm}(53,39)=\left(3^1\times 13^1\times 53^1=2067\right): \\
\end{array}$

Answer: lcm$(-53,-39)=2067$

\pagebreak

Determine if the following numbers are coprime to each other:

$416\text{ and }185$

\hrule

1. By definition, these numbers are coprime if their gcd is $1:$

$\gcd (416,185)=1$

2. Since these numbers have a gcd equal to $1,$ they are coprime:

Answer: 416 and $185$ are coprime

\end{document}
